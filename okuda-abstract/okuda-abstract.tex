\RequirePackage{plautopatch}
\plautopatchdisable{eso-pic} % https://oku.edu.mie-u.ac.jp/tex/mod/forum/discuss.php?d=2956, pdfpages との相性を良くする
\documentclass[12pt,dvipdfmx,uplatex]{jsarticle}
% \documentclass[11pt,autodetect-engine,dvi=dvipdfmx,ja=standard,
% label-section=modern, % モダーン!(Theorem Appendix A.3 みたいなアホがない)
% a4paper]{bxjsarticle}

\input{../../../preamble-j-a}
\input{../../../pikachu}
\input{../../../colorsforprint.tex}
\usepackage{pdfpages} % at the final compile, delete the option "demo"
\usepackage[top=10truemm,bottom=20truemm,left=18truemm,right=18truemm]{geometry} %DO NOT option ``a4paper", it changes the size of outputed PDF

\DeclareMathOperator{\ballsub}{\underset{\text{ball}}{\subset}}
\renewcommand{\dx}{\; dx}
\newcommand{\uah}{\underline{\ah}}
\newcommand{\MJ}{MJ}
\newcommand{\img}{\sqrt{-1}}
% \DeclareMathOperator{\MJ}{MJ}

\newcommand{\bigzero}{\mbox{\Large\textbf{0}}}
\newcommand{\rvline}{\hspace*{-\arraycolsep}\vline\hspace*{-\arraycolsep}}
\newcommand\Tstrut{\rule{0pt}{2.6ex}}         % = `top' strut
\newcommand\Bstrut{\rule[-0.9ex]{0pt}{0pt}}   % = `bottom' strut

\makeatletter
\newcommand{\neutralize}[1]{\expandafter\let\csname c@#1\endcsname\count@}
\makeatother

\newenvironment{claimbis}[1]
{\renewcommand{\thethm}{\ref{#1}$'$}%
  \neutralize{claim}\phantomsection
  \begin{claim}}
  {\end{claim}}

% \begin{align*}
%   \begin{array}{ccc}
%     M & \stackrel{\phi}{\longrightarrow} & M' \\
%     \rotatebox{90}{$\in$} & & \rotatebox{90}{$\in$} \\
%     p & \longmapsto & \phi(p)
%   \end{array}
% \end{align*}

% \begin{figure}[H]
%   \centering
%   \begin{tikzpicture}
%     \node (ef) at (0, 1.2) {$E\times F$};
%     \node (g) at (1.2, 1.2) {$G$};
%     \node (m) at (0, 0) {$M$};
%     \draw[->] (ef) to node[above] {$\scriptstyle b$} (g);
%     \draw[->] (ef) to node[left] {$\scriptstyle \phi$} (m);
%     \draw[->] (m) to node[right] {$\scriptstyle \exists !\tilde{b}$} (g);
%     \node (a) at (0.4, 0.7) {$\scriptstyle \circlearrowleft$};
%   \end{tikzpicture}
% \end{figure}

% \begin{empheq}[left={|x|=\empheqlbrace}]{alignat*=2}
%   x & \quad (x\geq 0) \\
%   -x & \quad (otherwise) 
% \end{empheq}

% easylist
% \ListProperties(Margin2=2em,Space1*=0em,FinalMark={)}, Style2*=\textbullet, Hide2=2)

% inline eqs. w/. numbering
% e.g.
% \inlineequation[eq:matsushima-p191-1]{\forall X\in\ge,\ \ad(X)\circ J = J\circ \ad(X)}

% \begin{figure}[H]
%   \centering
%   %   \raggedleft
%   %   \raggedrightp
%   %   \includegraphics[scale=0.08]{graph/fig1.jpg}
%   \includegraphics[scale=0.08]{graph/fig1-2.jpg}
%   \label{fig:fig1}
% \end{figure}


% % \maketitleの余白を調整
% \makeatletter
% \renewcommand{\@maketitle}{\newpage
%   % \null
%   % \vskip 2em 
% \begin{center}
%   {\LARGE \@title \par} % \vskip 1.5em
%   {\large \lineskip .5em
%   \begin{tabular}[t]{c}\@author
%   \end{tabular}\par
% }
%     %   \vskip 1em
%     %   {\large \@date}
% \end{center}
%   % \par
%   % \vskip 0.5em
% }
%   \makeatother

\usepackage{advdate} % \AdvanceDate[-1]\today の進捗報告,みたいな

% \setnewcounter{hoge}
% \newtheorem{theorem}{Theorem}[hoge]
\newcounter{countabst}
\setcounter{countabst}{0}
\newtheorem{yosou-a}[countabst]{予想}
\newtheorem{prob-a}[countabst]{問}
\newtheorem{nttdef-a}[countabst]{記号と定義}
\newtheorem{thm-a}[countabst]{定理}
\newtheorem{lem-a}[countabst]{補題}
\newtheorem{prop-a}[countabst]{命題}
\newtheorem{cor-a}[countabst]{系}
\newtheorem{def-a}[countabst]{定義}

\begin{document}

% \begin{titlepage}
% \newgeometry{top=10truemm,bottom=15truemm,left=12truemm,right=12truemm}
\huge
\centering
{\Huge 2021年度}

\vspace{4cm}

% \raggedright
\begin{center}
  修士論文題目
\end{center}

% \flushleft
\centering
\noindent
\underline{\large Riemann 対称空間上における測地線の簡約部分 Lie 代数への射影に対する有界性}

\underline{\large ---低階数・低次元の場合---}

\vspace{10cm}

\Large
\raggedright
\begin{tabular}{ll}
学生証番号 \quad {} & \underline{45-196010} \\
フリガナ & オクダ タカコ \\
氏名 & \underline{奥田 堯子}
\end{tabular}

\end{titlepage}
\begin{center}
  論文内容の要旨
\end{center}

\noindent\underline{修士論文題目}
\vspace{-1em}
\begin{center}
  Riemann対称空間上における測地線の簡約部分Lie代数への射影に対する有界性

  ---低階数・低次元の場合---
\end{center}
\vspace{-0.5em}

\noindent 氏名: 奥田 堯子

\vspace{0.5em}

本修士論文では,小林俊行氏による$\ha$射影の有界性に対する次の\Cref{prob:1121}について,$G$の実階数や$H$の次元が低い場合に肯定的な結果を得た ($\ha$射影の定義や記号は後述する).

\begin{prob-a}(小林俊行氏による)\label{prob:1121}
  $X\in \pe$に対し$Y(\real X)$が$ \ha\cap \pe$の有界な部分集合であることと「$X\in \per{\ha}\cap\pe $もしくは『$ [X_1, X_2]\neq 0 $かつ$\pe\cap\ze_{\ze(\ha)}(X) = 0$であること』」は同値であるか?

  ただし$X = X_1 + X_2 $はベクトル空間としての分解$\pe =(\pe\cap \ha)\oplus(\pe\cap\per{\ha}) $に対応する$X\in \pe$の分解とする.
\end{prob-a}
\vspace{-0.5em}
ここで$G$が実階数1のとき,「$X\in \per{\ha}\cap\pe $もしくは『$ [X_1, X_2]\neq 0 $かつ$\ze_{\ze(\ha)}(X) = 0$であること』」と$X\in \{0\}\cup\pe\setminus\ha $は同値である.


この論文の基本設定は以下の通りである.
\begin{nttdef-a}\label{nttdef:setting}
  \leavevmode\vspace{-1em}
  \begin{itemize}
  \item $G$を非コンパクト実簡約Lie 群,$H$は$G$の非コンパクトかつ連結成分有限個の閉部分群で,$G$のCartan対合$\Theta$に対して$H = \Theta H$を満たすものとする.
  \item $\ge \defeq \Lie G,\; \ha \defeq \Lie H$とし,$\ge = \ka\oplus \pe$を $\theta \defeq d\Theta$ によるCartan分解とする.
  \item  $e$を$G$の単位元とし,$o_K \defeq eK\in G/K$とする.
  \item $\lyama{-},{-} \ryama$を,$\ge$上の$G$-不変な非退化対称双線型形式で,「$\ka$上負定値,$\pe$上正定値で$\ka\perp \pe$」なるものとし,$\per{\ha}\ \defeq \{W\in \ge\mid \lyama W, \ha\ryama = \{0\}\} $とする.
  \end{itemize}  
\end{nttdef-a}
\vspace{-0.5em}

本修士論文の主題である$X\in \pe$の$\ha$射影$Y(X)\in \ha\cap \pe $は,次の\Cref{thm:kob89-lem6.1}により${(Y(X), Z(X))\defeq \inv{\pi}(e^X\cdot o_K)\in (\ha\cap\pe)\oplus (\per{\ha}\cap \pe)}$と定義される.
\begin{thm-a}(\cite[Lemma~6.1]{kob89}\label{thm:kob89-lem6.1})
  \Cref{nttdef:setting}の設定において$\pi\colon  (\ha\cap\pe)\oplus (\per{\ha}\cap \pe) \ni (Y, Z)\mapsto e^{Y}e^{Z}\cdot o_K \in G/K $は上への微分同相である.
\end{thm-a}

$e^{Y(X)}\cdot o_K$は「$e^{X}\cdot o_K$から$e^{\ha\cap\pe}\cdot o_K $に下ろした垂線の足」であり,$Y(\real X) $が有界であるか否かという問いは,幾何的には「$e^{tX}\cdot o_K$から$e^{\ha\cap\pe}\cdot o_K $に下ろした垂線の足全体の集合が有界であるか」という問いに対応する.

以下では$(G,H) $がどのような場合に,どのような証明方法でを示したかを具体的に述べる.

$G = \SU(1,2) $,$H= \SO(1,1)$の場合がトイモデルとなって$G$が実階数1の場合の\Cref{prob:1121}に対する肯定的な結果が得られた.

$G = \SU(1,2) $,$H= \SO(1,1)$の場合の証明は背理法による.例えば$X\in \pe\setminus\ha $に対して$Y(\real X) $が非有界,より具体的に$Y(t X) = s(t) Y$,$s(t) \to \infty$,$t\to \infty $なるとする.$G/K \simeq \{(z_1,z_2)\in \cpx^2 \mid \abs{z_1}^2 + \abs{z_2}^2 < 1 \} $であることを用いて$e^{Y(tX)}e^{Z(tX)}\cdot o_K $を計算すると,任意の$\epsilon > 0$に対して,ある$t_{\epsilon}\in \real$が存在して「$ e^{Y(t_{\epsilon} X)}e^{Z(t_{\epsilon} X)}\cdot o_K $と$o_K$を結ぶ測地線」が「$e^{Y(t_{\epsilon} X)}\cdot o_K $と$o_K$を結ぶ測地線」が$o_K$でなす角が$\epsilon $未満となる.これは$X$と$\ha\setminus\{0\} $のなす角度の最小値が非零であることに矛盾し,\Cref{prob:1121}と同値な「$X\in \{0\}\cup \pe\setminus\ha $であることと$ Y(\real X) $が有界であることが同値である」ということが証明できる.

これを踏まえて$G$が実階数1の実半単純Lie群,$\dim \ha\cap\pe =1 $の場合には次の命題を用いて\Cref{prob:1121}に対して肯定的な結果を得た.

\begin{prop-a}\label{prop:reduction}
  $G$を実階数1の実半単純Lie群とする.任意の$0\neq Y\in \ha\cap\pe $と任意の$X\in \pe\setminus \real Y$を固定したとき,$X,Y$を含む部分Lie環$\ge_0\subset \ge$で,$\ge_0\simeq \sulie(1,1) $あるいは$\ge_0\simeq \sulie(2,1)$なるものが存在する.また$\ge_0$の$G$における解析的部分群$G_0$は$G$の閉部分群である.
\end{prop-a}
\vspace{-0.5em}
\Cref{prop:reduction}は{$\SU(2,1) $-reduction},\cite{hel01}と\cite{yos38}の定理を併せて示される.

$G$が実階数1のLie群の積であり,$\ha\cap \pe $の各成分が1次元であるときも,成分ごとに見ることで$G$が実階数1の実半単純Lie群,$\dim \ha\cap\pe =1 $
の場合に帰着でき,\Cref{prob:1121}に対する肯定的な結果を得られる.

\Cref{prob:1121}の背景を説明するために\cite{ber88}の内容のごく一部を述べる.

まずいくつか用語を準備する.$G$を実簡約Lie群,$H$を$G$の閉部分群とし,$G/H$には左Haar測度$\mu_{G/H} $が存在すると仮定する.局所有界関数$r\colon G\to \real_{\geq 0} $がproperなradial functionであるとは,$r$が次の4条件を満たすことである.
\vspace{-1em}
\begin{enumerate}
\item \redtext{$e\in G$を単位元とするとき$r(e) = 0 $である.}\footnote{これは\cite{ber88}には明示されていません.}
\item 任意の$g\in G$に対し$r(g) = r(\inv{g})\geq 0  $である.
\item 任意の$g_1,g_2\in G$に対し$r(g_1g_2)\leq r(g_1) + r(g_2)  $である.
\item 任意の$R\geq 0$に対し,$B(R)\defeq \{g\in G\mid r(g)\leq R \} $は$G$の相対コンパクト集合である.
\end{enumerate}
\vspace{-0.5em}
properなradial function $r\colon G\to \real_{\geq 0} $から$r_{G/H}(gH)\defeq \inf_{h\in H}\{r(gh) \}$により定まる${r_{G/H}\colon G/H\to \real_{\geq 0}}$を$G/H$上のradial functionという.

$G/H$にはstandard measureと呼ばれる,次を満たす非自明なBorel 測度$m_X $が存在する.単位元のコンパクトな近傍で$B = \inv{B} $なる任意の$B\subset G$と任意の$g\in B$,$x\in G/H$に対し,ある定数$C_B\geq 0 $が存在して$g\cdot m_X \leq C_B m_X$,$ \inv{C_B} < m_X(Bx) < C_{B}$である.

$d = \inf\{d'\geq 0\mid \text{ある } C > 0\text{が存在して }  m_X(B(r))\leq C(1+r)^{d'}\} $であるとき,$G/H$のランクは$d$であると言う.

$G$の既約ユニタリ表現$V$が$G$の正則表現$L^2(G/H)$の既約分解に出現する必要条件は,非自明な$G$-絡作用素$\alpha_V\colon (C_c(G/H))^{\infty}\to V $が存在し,任意の$v\in V^{\infty} $,$d' > d$に対して$\dint_{G/H}\lbig|\beta_V(v)(x)(1+r(x))^{-d/2} \rbig|^2\; dx < \infty $なることである.ただし$\beta_V $は次の命題により$\alpha_V $と対応する$G$-絡作用素$\beta_V\colon V^{\infty}\to C(G/H)^{\infty}  $とする.
\begin{prop-a}(\cite[p.~678]{ber88})
  $G/H$の左Haar測度$\mu_{G/H} $を1つ固定する.次の同型写像が存在する.$\Hom_{G}((C_c(G/H))^{\infty}, V )\to \Hom_{G}(V^{\infty}, C(G/H)^{\infty}) $,$\alpha_V\mapsto \beta_V$ただし任意の$v\in V$,$\phi \in  (C_c(G/H))^{\infty} $に対し$ \lyama v,\alpha_V(\phi)\ryama_{V} = \dint_{G/H}\beta_V(v)\phi d\mu_X  $である.
\end{prop-a}

$G$を実簡約Lie群,$H$を$G$の閉部分群とし,ある可換部分群$B\subset G$が存在して$G = KBH $というCartan分解を持つとする.任意の$X\in \Lie B$に対して$Y(\real X) $が有界であれば,$G/H$がランク$d \defeq \dim B $となるための条件のうちの1つである,「ある定数$C > 0$が存在して,任意の$X\in \Lie B$に対し,$d_{G/K}(\exp(-X)\cdot o_K, o_K )  - \inf\{d_{G/K}(\exp(-X)\cdot o_K,h\cdot o_K ) \mid h\in H\}\leq C$であること」が満たされる.

以上が本修士論文の表現論的な背景である.


\clearpage
\begin{thebibliography}{99}
  \vspace{-0.5em}
\bibitem[Ber88]{ber88} J.~N.~Bernstein, \textit{On the support of Plancherel measure}, J.~Geom.~Phys., Vol. 5, n. 4, 1988, pp. 663--710.
% \bibitem[BH99]{bh99} M.~R.~Bridson and A.~Haefliger, Metric Spaces of Non-Positive Curvature, Grundlehren der mathematischen Wissensschaften, Vol.~319, Springer, 1999.
% \bibitem[Ebe72a]{e72-1} P.~Eberlien, \textit{Geodesic Flows on Negatively Curved Manifolds I}, Ann.~of~Math.~(2), Vol.~95, 1972, pp.~492--510.
% \bibitem[Ebe72b]{e72-2} P.~Eberlien, \textit{Geodesic Flow in Certain Manifolds without Conjugate Points}, Trans.~Amer.~Math.~Soc., Vol.~167, 1972, pp.~151--70.
\bibitem[Hel01]{hel01} S. Helgason, Differential Geometry, Lie Groups, and Symmetric Spaces, GSM, Vol. 34, AMS, 2001.
\bibitem[KK16]{kk16} F. Kassel and T. Kobayashi, \textit{{\Poincare} series for non-Riemannian locally symmetric spaces}, Adv. Math., Vol. 287, 2016, pp. 123--236.
\bibitem[Kob89]{kob89} T.~Kobayashi, 
  \textit{Proper action on a homogeneous space of reductive type},
  Math.~Ann., Vol. 285, Issue. 2, 1989, pp. 249--263.
\bibitem[小林95]{kobayashi95} 小林俊行,球等質多様体上の調和解析入門,第3回整数論サマースクール `等質空間と保型形式' 所収,佐藤文広 編,長野,1995,pp. 22-41.
\bibitem[Kob97]{kob97} T.~Kobayashi, \textit{Invariant mesures on homogeneous manifolds of reductive type}, J.~Reine~Angew.~Math., Vol. 1997, No. 490--1, 1997, pp. 37--54.
\bibitem[Yos38]{yos38} K.~Yosida, \textit{A Theorem concerning the Semi-Simple Lie Groups}, \\Tohoku~Mathematical~Journal, First Series, Vol.~44, 1938, pp.~81--84.
\end{thebibliography}


\begin{comment}
  \begin{def-a}{\cite[Definition~1.3]{e72-1}}\label{def:visibility}

    $M$が完備かつ非正曲率をもつ1-連結Riemann多様体であるとき,$M$をHadamard多様体といい,Hadamard多様体$M$が visibility manifold であるとは,任意の$ p\in M$と任意の$ \epsilon > 0$に対し,ある$r(p,\epsilon) >0 $が存在して,測地線$\gamma\colon [t_0, t_1]\to X $が$d_{M}(p, \gamma(t))\geq r(p,\epsilon) $,$t\in [t_0, t_1]$ならば,$\measuredangle_{p}(\gamma(t_0), \gamma(t_1)) \leq \epsilon $であることである.
  \end{def-a}

  \bluetext{$M$がvisibility manifold であるとは,幾何的に見れば}\footnote{図をつけようと思っています.2022/01/10}

  \begin{thm-a}{\cite[p.~296, 9.33~Theorem]{bh99}, originally \cite[Theorem~4.1]{e72-2}}\label{thm:visibility-and-rank}
    
    ある$C\cptsub M$が存在して$ M = \bigcup\{f(C)\mid f\in \isom(M) \}  $なるHadamard多様体$M$に対し,次は同値である.
    \vspace{-1em}
    \begin{enumerate}
      \renewcommand{\labelenumi}{(\roman{enumi})}
    \item $M$はvisibility manifoldである.
    \item 全測地的な部分Riemann多様体$M'\subset M$で$\real^2$と等長同型なものが存在しない.
    \end{enumerate}
  \end{thm-a}

  ここでRiemann対称空間はHadamard多様体であり,\Cref{thm:visibility-and-rank}の (ii) は$G$の実階数が1以下であることと同値である.したがって$G$の実階数が1の場合$G/K$はvisibility manifoldであり,$G = \SU(1,2) $,$H= \SO(1,1)$の場合の証明と全く同様にして背理法により\Cref{prob:1121}が示される.
\end{comment}
\end{document}
