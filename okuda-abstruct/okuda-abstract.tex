\RequirePackage{plautopatch}
\plautopatchdisable{eso-pic} % https://oku.edu.mie-u.ac.jp/tex/mod/forum/discuss.php?d=2956, pdfpages との相性を良くする
\documentclass[12pt,dvipdfmx,uplatex]{jsarticle}
% \documentclass[11pt,autodetect-engine,dvi=dvipdfmx,ja=standard,
% label-section=modern, % モダーン!(Theorem Appendix A.3 みたいなアホがない)
% a4paper]{bxjsarticle}

\input{../../../preamble-j}
\input{../../../pikachu}
\input{../../../colorsforprint.tex}
\usepackage{pdfpages} % at the final compile, delete the option "demo"
\usepackage[top=10truemm,bottom=20truemm,left=20truemm,right=20truemm]{geometry} %DO NOT option ``a4paper", it changes the size of outputed PDF

\DeclareMathOperator{\ballsub}{\underset{\text{ball}}{\subset}}
\renewcommand{\dx}{\; dx}
\newcommand{\uah}{\underline{\ah}}
\newcommand{\MJ}{MJ}
\newcommand{\img}{\sqrt{-1}}
% \DeclareMathOperator{\MJ}{MJ}

\newcommand{\bigzero}{\mbox{\Large\textbf{0}}}
\newcommand{\rvline}{\hspace*{-\arraycolsep}\vline\hspace*{-\arraycolsep}}
\newcommand\Tstrut{\rule{0pt}{2.6ex}}         % = `top' strut
\newcommand\Bstrut{\rule[-0.9ex]{0pt}{0pt}}   % = `bottom' strut

\makeatletter
\newcommand{\neutralize}[1]{\expandafter\let\csname c@#1\endcsname\count@}
\makeatother

\newenvironment{claimbis}[1]
  {\renewcommand{\thethm}{\ref{#1}$'$}%
   \neutralize{claim}\phantomsection
   \begin{claim}}
  {\end{claim}}

% \begin{align*}
%   \begin{array}{ccc}
%     M & \stackrel{\phi}{\longrightarrow} & M' \\
%     \rotatebox{90}{$\in$} & & \rotatebox{90}{$\in$} \\
%     p & \longmapsto & \phi(p)
%   \end{array}
% \end{align*}

% \begin{figure}[H]
%   \centering
%   \begin{tikzpicture}
%     \node (ef) at (0, 1.2) {$E\times F$};
%     \node (g) at (1.2, 1.2) {$G$};
%     \node (m) at (0, 0) {$M$};
%     \draw[->] (ef) to node[above] {$\scriptstyle b$} (g);
%     \draw[->] (ef) to node[left] {$\scriptstyle \phi$} (m);
%     \draw[->] (m) to node[right] {$\scriptstyle \exists !\tilde{b}$} (g);
%     \node (a) at (0.4, 0.7) {$\scriptstyle \circlearrowleft$};
%   \end{tikzpicture}
% \end{figure}

% \begin{empheq}[left={|x|=\empheqlbrace}]{alignat*=2}
%   x & \quad (x\geq 0) \\
%   -x & \quad (otherwise) 
% \end{empheq}

% easylist
% \ListProperties(Margin2=2em,Space1*=0em,FinalMark={)}, Style2*=\textbullet, Hide2=2)

% inline eqs. w/. numbering
% e.g.
% \inlineequation[eq:matsushima-p191-1]{\forall X\in\ge,\ \ad(X)\circ J = J\circ \ad(X)}

% \begin{figure}[H]
%   \centering
%   %   \raggedleft
%   %   \raggedrightp
%   %   \includegraphics[scale=0.08]{graph/fig1.jpg}
%   \includegraphics[scale=0.08]{graph/fig1-2.jpg}
%   \label{fig:fig1}
% \end{figure}


% % \maketitleの余白を調整
% \makeatletter
% \renewcommand{\@maketitle}{\newpage
%   % \null
%   % \vskip 2em 
%   \begin{center}
%     {\LARGE \@title \par} % \vskip 1.5em
%     {\large \lineskip .5em
%       \begin{tabular}[t]{c}\@author
%       \end{tabular}\par
%     }
%     % \vskip 1em
%     % {\large \@date}
%   \end{center}
%   % \par
%   % \vskip 0.5em
% }
% \makeatother

\usepackage{advdate} % \AdvanceDate[-1]\today の進捗報告,みたいな

% \setnewcounter{hoge}
% \newtheorem{theorem}{Theorem}[hoge]
\newcounter{countabst}
\setcounter{countabst}{0}
\newtheorem{yosou-a}[countabst]{予想}
\newtheorem{nttdef-a}[countabst]{記号と定義}
\newtheorem{thm-a}[countabst]{定理}
% \newenvironment{yosou-a}[1]{\noindent \textbf{予想 #1}  \par}{}

% \newenvironment{nttdef-a}[1]{\noindent \textbf{記号と定義 #1}  \par}{}

\begin{document}

% \begin{titlepage}
% \newgeometry{top=10truemm,bottom=15truemm,left=12truemm,right=12truemm}
\huge
\centering
{\Huge 2021年度}

\vspace{4cm}

% \raggedright
\begin{center}
  修士論文題目
\end{center}

% \flushleft
\centering
\noindent
\underline{\large Riemann 対称空間上における測地線の簡約部分 Lie 代数への射影に対する有界性}

\underline{\large ---低階数・低次元の場合---}

\vspace{10cm}

\Large
\raggedright
\begin{tabular}{ll}
学生証番号 \quad {} & \underline{45-196010} \\
フリガナ & オクダ タカコ \\
氏名 & \underline{奥田 堯子}
\end{tabular}

\end{titlepage}
\begin{center}
論文内容の要旨
\end{center}

\noindent\underline{修士論文題目}

\begin{center}
Riemann 対称空間上における測地線の簡約部分 Lie 代数への射影に対する有界性

---低階数・低次元の場合---
\end{center}

\noindent 氏名: 奥田 堯子

\vspace{1em}

本修士論文では,小林俊行氏による次の\Cref{yosou:1121}を$G$の実階数や$H$の次元が低い場合に証明した (記号は後述する).

\begin{yosou-a}\label{yosou:1121}
  
  ベクトル空間としての分解$\pe =(\pe\cap \ha)\oplus(\pe\cap\per{\ha}) $に沿って$X = X_1 + X_2 $と分解すると,$\pe_{H,\bdd} = \{X\in \pe\mid [X_1, X_2]\neq 0 \text{ or } X_1 = 0 \}$である.
\end{yosou-a}

慣例的な記号や設定は以下の通りとする.
\vspace{-1em}
\begin{nttdef*}\textcolor{white}{hoge}
  \vspace{-1em}
  
  \begin{itemize}
  \item $G$を非コンパクト実半単純Lie 群,$H$を$G$のCartan対合$\Theta$に対する非コンパクトな簡約部分Lie群とする.
  \item $\ge \defeq \Lie G,\; \ha \defeq \Lie H$とし,$\ge = \ka\oplus \pe$を $\theta \defeq d\Theta$ によるCartan分解とする.
  \item  $e_G$を$G$の単位元とし,$o_K \defeq e_GK\in G/K$とする.
  \item $B({-}, {-}) $を$\ge$のKilling形式とし,$\per{\ha}\cap \pe \defeq \{W\in \pe\mid B(Y, W) = 0 ,\forall Y\in \ha\cap \pe\} $とする.
  \end{itemize}
  
\end{nttdef*}

以下の \Cref{thm:kob89-lem6.1} を用いて,$X\in \pe$に対し,$(Y(X), Z(X))\defeq \inv{\pi}(e^X\cdot o_K)\in (\ha\cap\pe)\oplus (\per{\ha}\cap \pe)$と定義する.
\begin{thm-a}\cite[Lemma~6.1]{kob89}\label{thm:kob89-lem6.1}

  $\pi\colon  (\ha\cap\pe)\oplus (\per{\ha}\cap \pe) \ni (Y, Z)\mapsto e^{Y}e^{Z}\cdot o_K \in G/K $は上への微分同相である.
\end{thm-a}



\begin{thebibliography}{99}
\bibitem[Kob89]{kob89} T.~Kobayashi, 
  \textit{Proper action on a homogeneous space of reductive type},
  Math.~Ann., Vol. 285, Issue. 2, 1989, pp. 249--263.  
\bibitem[Kob97]{kob97} T.~Kobayashi, \textit{Invariant mesures on homogeneous manifolds of reductive type}, J.~Reine~Angew.~Math., Vol. 1997, No. 490--1, 1997, pp. 37--54

\end{thebibliography}

\end{document}
