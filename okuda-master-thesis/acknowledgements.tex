本研究および修士課程全体において常に洞察に富むご助言と丁寧なご指導を賜った指導教員の小林俊行教授に深謝の意を表する.% 小林先生のご指導なしには本論文はあり得なかった.
また,文献の情報から数学的な議論にわたり様々なご助言をくださった,修了生も含む小林研究室のみなさまにも心より感謝する.特に小林研究室の田内大渡氏には幾度か議論いただいたことに御礼申し上げたい.

最後に,学部時代からセミナーに付き合ってくださり,ときに精神的にも支えてくださった友人と家族に感謝の意を表して謝辞とする.

\begin{comment}
  多くのご指導とご協力を賜りました小林俊行先生に深謝の意を表します.粘り強く丁寧なご指導のおかげで本論文は完成いたしました.
  また,論文執筆の際にご助言や相談に乗っていただいた,研究室の各位そして同期の皆様にも感謝いたします.
  最後に,温かく見守り支援してくださった友人と家族に感謝の意を表して謝辞といたします.




  First of all, I would like to show my profound gratitude to professor Toshiyuki Kobayashi, for his constant support and encouragement, who introduced me to this beautiful field. His strict and patient guidance broadened my knowledge, making me grow up as a mathematician. This master dissertation has been possible thanks to him.
  I am also deeply thankful to all the members of Kobayashi’s lab; specially to K. Kannaka, T. Satomi, Y. Inoue, R. Fujita and T. Okuda, who are always supporting me and keeping me active with interesting questions and debates.
  The support of the office members of the faculty was also remarkable, specially the case of A. Nakamura; who constantly encouraged me these last two years.
  I would also like to express my appreciation to I. Vald ́es S ́anchez, J. M. Gonz ́alez Vega, S. Ishii and R. S ́anchez Molero for their continuous encouragement and understanding; they are always keeping me motivated.
  Last but not least, special thanks to the Japanese Ministry of Education, Culture, Sports, Science and Technology (MEXT), for their financial support throughout the last two years.
\end{comment}