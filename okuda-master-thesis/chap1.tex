%%% Local Variables:
%%% mode: japanese-latex
%%% TeX-engine: uptex
%%% TeX-master: "okuda-master-thesis"
%%% TeX-PDF-mode: t
%%% TeX-PDF-from-DVI: "Dvipdfmx"
%%% End:

\subsection{記号の設定}
本論文の基本的な設定は次のとおりであり,この他に必要な条件は都度明示することとする.

\begin{nttdef}\textcolor{white}{hoge}
  \vspace{-1em}
  \begin{itemize}
  \item $G$を非コンパクト実半単純Lie 群,$H$を$G$の部分Lie群で,$G$のCartan対合$\Theta$に対して$\Theta H = H$なるものとする.
  \item $\ge \defeq \Lie G,\; \ha \defeq \Lie H$とし,$\ge = \ka\oplus \pe$を $\theta \defeq d\Theta$ によるCartan分解とする.
  \item  $e$を$G$の単位元とし,$o_K \defeq eK\in G/K$とする.
  \item $B({-}, {-}) $を$\ge$のKilling形式とし,$\per{\ha}\cap\pe \defeq \{W\in \pe\mid \text{任意の} Y\in \ha\cap\pe \text{に対して}  B(Y, W) = 0\} $とする.
  % \item 
  \end{itemize}  
\end{nttdef}

以下の\Cref{thm:kob89-lem6.1}を用いて,$X\in \pe$に対し,$(Y(X), Z(X))\defeq \inv{\pi}(e^X\cdot o_K)\in (\ha\cap\pe)\oplus (\per{\ha}\cap \pe)$と定義する.
\begin{thm}\cite[Lemma~6.1]{kob89}\label{thm:kob89-lem6.1}

  $\pi\colon  (\ha\cap\pe)\oplus (\per{\ha}\cap \pe) \ni (Y, Z)\mapsto e^{Y}e^{Z}\cdot o_K \in G/K $は上への微分同相である.
\end{thm}

ここで,$Y(\real X) $の有界性について,次の予想が小林俊行氏によって立てられた.

\begin{yosou}(by T.~Kobayashi)\label{yosou:1121}
  
  ベクトル空間としての分解$\pe =(\pe\cap \ha)\oplus(\pe\cap\per{\ha}) $に対応して$X = X_1 + X_2 $と分解すると,$\pe_{H,\bdd} = \{X\in \pe\mid [X_1, X_2]\neq 0 \text{ あるいは } X_1 = 0 \}$である.
\end{yosou}

\Cref{yosou:1121}についての基本的な事項を挙げる.

\begin{enumerate}
\item $\pe_{H,\bdd} \subset \{X\in \pe\mid [X_1, X_2]\neq 0 \text{ あるいは } X_1 = 0 \}$である.
\item 
\end{enumerate}

ここで,$Z(\real X) $の有界性については次の定理が知られている.

\begin{thm}\cite[Lemmma~5.4]{kob97}\label{thm:kob97}
  
  $X\in \pe$に対し,$\norm{Z(X)}\geq \norm{X} \sin\phi(X, \ha\cap\pe)$である.

  ここに$\phi(X,\ha\cap \pe) $は$X$と$\ha\cap \pe$の元がなす角度の最小値$0\leq \phi(X,\ha\cap \pe) \leq \frac{\pi}{2} $であり,
  $X\in \pe\setminus \ha \iff \phi(X,\ha\cap \pe)\neq 0 $である.
\end{thm}
つまり$ X\in \pe\setminus \ha$ならば$\norm{Z(t X)}\to \infty $,$\abs{t}\to \infty $である.



