%%% Local Variables:
%%% mode: japanese-latex
%%% TeX-engine: uptex
%%% TeX-master: "okuda-master-thesis"
%%% TeX-PDF-mode: t
%%% TeX-PDF-from-DVI: "Dvipdfmx"
%%% End:

\subsection{記号の設定}
本論文の基本的な設定は次のとおりであり,この他に必要な条件は都度明示することとする.

\begin{nttdef}\textcolor{white}{hoge}
  \vspace{-1em}
  \begin{itemize}
  \item $G$を非コンパクト実半単純Lie 群,$H$を$G$の非コンパクトな部分Lie群で,$G$のCartan対合$\Theta$に対して$\Theta H = H$なるものとする.
  \item $\ge \defeq \Lie G,\; \ha \defeq \Lie H$とし,$\ge = \ka\oplus \pe$を $\theta \defeq d\Theta$ によるCartan分解とする.
  \item  $e$を$G$の単位元とし,$o_K \defeq eK\in G/K$とする.
  \item $B({-}, {-}) $を$\ge$のKilling形式とし,$\per{\ha}\cap\pe \defeq \{W\in \pe\mid \text{任意の} Y\in \ha\cap\pe \text{に対して}  B(Y, W) = 0\} $とする.
  % \item 
  \end{itemize}  
\end{nttdef}

以下の\Cref{thm:kob89-lem6.1}を用いて,$X\in \pe$に対し,$(Y(X), Z(X))\defeq \inv{\pi}(e^X\cdot o_K)\in (\ha\cap\pe)\oplus (\per{\ha}\cap \pe)$と定義する.
\begin{thm}\cite[Lemma~6.1]{kob89}\label{thm:kob89-lem6.1}

  $\pi\colon  (\ha\cap\pe)\oplus (\per{\ha}\cap \pe) \ni (Y, Z)\mapsto e^{Y}e^{Z}\cdot o_K \in G/K $は上への微分同相である.
\end{thm}


ここで,$Y(\real X) $の有界性について,次の予想が小林俊行氏によって立てられた.

\begin{yosou}(by T.~Kobayashi)\label{yosou:1121}
  
  ベクトル空間としての分解$\pe =(\pe\cap \ha)\oplus(\pe\cap\per{\ha}) $に対応して$X = X_1 + X_2 $と分解すると,$\pe_{H,\bdd} = \{X\in \pe\mid [X_1, X_2]\neq 0 \text{ あるいは } X_1 = 0 \}$である.
\end{yosou}

\Cref{yosou:1121}についての基本的な事項を挙げる.

\begin{lem}\label{lem:basic-yosou}
  \leavevmode\vspace{-1em}
  \begin{enumerate}
  \item $\pe_{H,\bdd} \subset \{X\in \pe\mid [X_1, X_2]\neq 0 \text{ あるいは } X_1 = 0 \}$である.\bluetext{もっと書くことはあるはず.2022/01/11}
  \item $X \in \pe $が$X_1 = 0$を満たすならば$X\in \pe_{H,\bdd} $である.
  \item 1,2より\Cref{yosou:1121}と「$X\in \pe$が$[X_1,X_2]\neq 0$ならば$X\in \pe_{H,\bdd} $である」は同値である.
  \item $G$が実階数1のとき,\Cref{yosou:1121}と「$\pe_{H,\bdd} =  \{0\}\cup \pe\setminus\ha $」は同値である.
  \end{enumerate}
\end{lem}

\begin{pfwn}{\Cref{lem:basic-yosou}}
  \leavevmode
  \vspace{-2em}
  \begin{enumerate}
  \item 背理法による.$[X_1,X_2 ] = 0$かつ$X_1\neq 0$なる$X  \in \pe $に対しては$[X_1,X_2 ] = 0$より$e^{tX_1}e^{tX_2}\cdot o_K = e^{t(X_1 + X_2)}\cdot o_K = e^{tX}\cdot o_K$であり,$Y(tX) = tX_1 $,$Z(tX) = tX_2 $であることから$Y(\real X) = \real X_1 $となり,$X_1\neq 0$より$Y(\real X)$は有界集合とならない.
  \item $X_1 = 0\iff X\in \per{\ha}\cap\pe $より$Z(tX) = tX $,$Y(tX) = 0 $であることによる.
  \item[4.] 対偶を示す.$X\in \pe$に対し,$[X_1,X_2] = 0 $かつ$X_1 \neq 0\iff X \in \ha\setminus\{0\} $を示せば良い.$G$の実階数は1で,$H$は非コンパクトであるから,$\ha\subset \pe$であり,$\ha$は$\ge$の極大可換部分空間である.よって$X_1\neq 0$かつ$[X_1,X_2] = 0  \implies X_2 = 0 $であり,$X = X_1 + X_2\in \ha\setminus\{0\} $を得る.
  \end{enumerate}
  
\end{pfwn}

$Y(\real X) $の有界性は$\Ad(k) $-不変である;
\begin{lem}\label{lem:1101}
  $k\in K$,$X\in \pe$に対し,$X'\defeq \Ad(k)X $,$\ha'\defeq \Ad(k)\ha $とするとき,$Y(\real X)$が有界$\iff Y'(\real X') $が有界である.

  ここで$Y'(X'), Z'(X') $を,微分同相$\pi'\colon (\ha'\cap \pe)\oplus (\per{\ha'}\cap \pe)\ni (Y',Z')\mapsto e^{Y'}e^{Z'}\cdot o_K  $を用いて,$X'\in \pe$に対し,$(Y'(X'), Z'(X')) = \inv{\pi'}(e^{X'}\cdot o_K) $と定める.
\end{lem}

\begin{pfwn}{\Cref{lem:1101}}

  主張は$(X,\ha) $と$(X',\ha')$に対して対称的であるから,$Y(\real X) $が有界$\Rightarrow Y'(\real X') $が有界,のみを示せば十分である.

  任意に$r\in \real$を取る.$e^{rX'}\cdot o_K = e^{Y'(r X')}e^{Z'(r X')}\cdot o_K  $であり,両辺に左から$\inv{k} $を掛けると,$e^{r X} = e^{\Ad(\inv{k})( Y'(r X'))}e^{\Ad(\inv{k})( Z'(r X'))}\cdot o_K  $を得る.ここで$Y'(rX')\in \ha'\cap \pe $,$Z'(r X')\in \per{\ha'}\cap \pe $であるから$\Ad(\inv{k})(Y'(r X'))\in \ha\cap \pe $,$\Ad(\inv{k})(Z'(r X')) \in \per{\ha}\cap \pe $である.

  \Cref{thm:kob89-lem6.1}により$\pi$は微分同相であるから任意の$r\in \real$に対して$\Ad(\inv{k})(Y'(r X')) = Y(rX)  $であるから,$Y'(\real X) = \Ad(k)(Y(\real X))  $であり,$\Ad(k) $は有限次元空間の間の線型写像であるから有界性を保つ.

  以上から\Cref{lem:1101}が示された.
  
\end{pfwn}


$Z(\real X) $の有界性については次の定理が知られており,有界性の判定はLie環の言葉のみで行える.

\begin{thm}\cite[Lemmma~5.4]{kob97}\label{thm:kob97}
  
  $X\in \pe$に対し,$\norm{Z(X)}\geq \norm{X} \sin\phi(X, \ha\cap\pe)$である.

  ここに$\phi(X,\ha\cap \pe) $は$X$と$\ha\cap \pe$の元がなす角度の最小値$0\leq \phi(X,\ha\cap \pe) \leq \frac{\pi}{2} $であり,
  $X\in \pe\setminus \ha \iff \phi(X,\ha\cap \pe)\neq 0 $である.
\end{thm}

つまり$ X\in \pe\setminus \ha$ならば$\norm{Z(t X)}\to \infty $,$\abs{t}\to \infty $である.


\subsection{\Cref{yosou:1121}の観察: $G = \SU(1,1) $,$H = \SO(1,1) $の場合}

$G = \SU(1,1) $,$H = \SO(1,1) \defeq\lbig\{
\begin{pmatrix}
  \cosh t & \sinh t\\ \sinh t & \cosh t
\end{pmatrix}
\relmiddle| t\in \real \rbig\} $の場合に\Cref{yosou:1121}が正しいことは直接計算により確かめられる.

\begin{prop}\label{prop:yosou-eg}
  $G = \SU(1,1) $,$H = \SO(1,1) $のとき\Cref{yosou:1121}は正しい.
\end{prop}

\bluetext{$\sulie(1,1) $のKilling形式と$r = \tanh t$の関係を明記せよ.}

\begin{lem}\label{lem:riem-metric-su(1,1)}
  $\sulie(1,1)$のKilling形式から定まる{\Poincare}円板$G/K$の計量は
\end{lem}

% \begin{lem}\label{lem:1018}
%   $W\in \pe$に対し,$e^W \cdot o_K $が$o_K$から Euclid 距離で$\tanh p$,$p \geq 0$の位置にある場合,$\norm{W} = \frac{p}{2} $である.
% \end{lem}


\begin{pfwn}{\Cref{prop:yosou-eg}}

  $k_{\theta} \defeq \diag(e^{\sqrt{-1}\theta},e^{-\sqrt{-1}\theta}) $,$X_{\theta} \defeq k_{\theta/2}
  \begin{pmatrix}
    0 & 1 \\ 1 & 0
  \end{pmatrix}
  k_{-\theta/2}$とすると,$\pe\setminus\{0\} =  \{tX_{\theta}\mid t\in \real_{>0},\ 0\leq \theta\leq \pi\}$である.この$X_{\theta} $と$t\in \real$に対して$Y(tX_{\theta} ) = s
  \begin{pmatrix}
    0 & 1 \\ 1 & 0
  \end{pmatrix}
  $なる$s\in \real $を求める.


   
  \begin{figure}[H]
    \centering
    % \raggedleft
    % \raggedrightp
    \includegraphics[scale=0.08]{../graph/yosou-eg-1.jpg}
    % \includegraphics[scale=0.3]{../graph/y-and-z.pdf}
    \caption{}
    \label{fig:yosou-eg-1}
  \end{figure}

  右の円の Euclid 距離での半径を$R$とし,$e^{tX_{\theta}}\cdot o_K $から$H\cdot o_K$への垂線の足の$o_K$からの Euclid 距離を$h$とするとき,外側の青色の直角三角形に対して三平方の定理を用いて$(h+R)^2 = R^2 +  1 $より$R = \frac{1-h^2}{2h} , R+h = \frac{1+h^2}{2h}  $を得る.

  さらに下の紫色の三角形に対して余弦定理を用いて$R^2 = (R+h)^2 + r^2 - 2(R+h) \cos\theta  $を得,$\inlineequation[eq:1018-main]{\dfrac{2r\cos\theta}{r^2 + 1} = \dfrac{2h}{h^2 + 1} }$を得る.

  ここで$r = \tanh t$,$h = \tanh s$であり\Cref{eq:1018-main} は$\cos\theta \tanh 2t = \tanh 2s $と書き直せる.したがって$X_{\theta}$に対して$Y(\real X) $が有界$\iff \abs{\cos\theta}\neq 1 \iff  X\nin \ha  $である.


\end{pfwn}

\begin{cor}\label{cor:yosou-eg}
  $G = \SO(1,n) $,$H = \SO(1,k) $,$1\leq k\leq n-1$に対して\Cref{yosou:1121}は正しい.
\end{cor}


\begin{pfwn}{\Cref{cor:yosou-eg}}
  「$e^X\cdot o_K $と$o_K$を結ぶ直線」と$H\cdot o_K$で張られる超平面で {\Poincare}球$\SO(1,n)/\SO(n)$を切った際の断面を考える.
  \begin{figure}[H]
    \centering
    % \raggedleft
    % \raggedrightp
    % \includegraphics[scale=0.08]{../graph/fig1.jpg}
    \includegraphics[scale=0.1]{../graph/son1.jpg}
    \caption{}
    \label{fig:son1}
  \end{figure}
  
  この断面に現れるのは\Cref{fig:yosou-eg-1}と同じであるから,同様の計算により\Cref{cor:yosou-eg}を得る.
  
\end{pfwn}