\subsection{具体例: 実階数1の古典型単純Lie群}
\begin{prop}\label{prop:classical-rank-one}
  $G = \SO(1,n)$,$ \SU(1, n)$,$\Sp(1,n) $,$H = \SO(1,1) $,$n\geq 2$に対して\Cref{yosou:1121} は正しい.
\end{prop}

$G = \Sp(1,2) $,$\ha = \real \begin{pmatrix}
    0 & 1 & 0 \\
    1 & 0 & 0\\
    0 & 0 & 0
  \end{pmatrix}$の場合にのみ示す.その他の場合も全く同様の議論である.
\begin{prop}\label{prop:1127-main}
  $G = \Sp(1,2) $,$H = \SO(1,1)$,$X\in \pe$に対し,$Y(\real X) $が有界$\iff X\in \pe\setminus \ha  $ or $X = 0$である.
\end{prop}

ただし,$H$は$G$の左上に入っている.すなわち,$\Lie H = \ha = \real A $,$A\defeq \begin{pmatrix}
  0 & 1 & 0 \\
  1 & 0 & 0\\
  0 & 0 & 0
\end{pmatrix}$とする.

\begin{nttdef}
  
  $\quat$を四元数体とする.$\Sp(1,2)/\Sp(1)\times \Sp(2) \simeq \{(z_1, z_2)\mid z_1,z_2\in \quat ,\; \abs{z_1}^2 + \abs{z_2}^2   < 1 \} =: \quat\mathbb{H}^2 $である.これは自然表現$\Sp(1,2)\leftaction \quat^2 $の$\trans{(1,0,0)} $軌道を考え,第2,第3成分に第1成分の逆数を右からかけた空間が$\quat\mathbb{H}^2$と微分同相であるためであり,$\Sp(1,2)\leftaction \quat^3 $の$\trans{(1,0,0)} $軌道の点$
  \begin{pmatrix}
    z_0 \\ z_1 \\ z_2 
  \end{pmatrix}
  $に対応する$\quat\mathbb{H}^2$の点を$
  \lbig[ \begin{pmatrix}
    z_0 \\ z_1 \\ z_2 
  \end{pmatrix}\rbig] = \lbig[ \begin{pmatrix}
    1 \\ z_1\inv{z_0} \\ z_2\inv{z_0} 
  \end{pmatrix}\rbig] 
  $と書く.
\end{nttdef}

愚直な行列計算により,次が示される.
\begin{lem}\label{lem:exp-quat}
  
  $\forall z,w\in \quat$に対し,$\exp
  \begin{pmatrix}
    0 & z & w  \\
    \bar{z} & 0 & 0\\
    \bar{w} & 0 & 0
  \end{pmatrix}
  =
  \begin{pmatrix}
    \cosh r &  \ast & \ast \\
    \\
    \dfrac{\bar{z}}{r} \sinh r &  \ast & \ast \\
    \\
    \dfrac{\bar{w}}{r}\sinh r &  \ast & \ast 
  \end{pmatrix}
  $,ただし$r \defeq \sqrt{\abs{z}^2 + \abs{w}^2 } $,である.
\end{lem}

\begin{pfwn}{\Cref{prop:1127-main}}
  
  % $\Lie H = \ha = \real A $,$A\defeq \begin{pmatrix}
  %   0 & 1 & 0 \\
  %   1 & 0 & 0\\
  %   0 & 0 & 0
  % \end{pmatrix}$とする.
  $X = 0\Rightarrow Y(\real X) = \{0\} $と$X\in \ha\setminus\{0\} $のときに$Y(\real X) $が非有界であることは明らかであるから,$X\nin \ha $の場合にのみ議論すればよい.% つまり
  % $X =
  % \begin{pmatrix}
  %   0 & z & w \\
  %   \bar{z} & 0 & 0 \\
  %   \bar{w} & 0 & 0 
  % \end{pmatrix}
  % \in \pe\setminus\ha $,$z,w\in \quat \st \abs{z}^2 +\abs{w}^2  = 1 $を任意に1つ固定して議論して一般性を失わない.このとき,$X\in\pe\setminus \ha $より$\re z \neq \pm 1$であることに注意する ($\re\colon \quat\ni a+bi+cj+dk\mapsto a\in \real$とする).

  $G$の Cartan 対合を$\Theta(g) = \inv{(g^{*})} $ ($g^{*}$は$g$の共役転置) とするとき,$\Theta(e^{Y(tX)}e^{Z(tX)})\cdot o_K = e^{-Y(tX)}e^{-Z(tX)}\cdot o_K = \Theta(e^{X})\cdot o_K = e^{-X}\cdot o_K $より,「$Y(\real X) $が非有界$\iff Y(\real X)\subset \real A $が上に非有界」である.

  したがって,$Y (\real X) $が非有界であるとき,列$\{t_n \in \real \}_{n\in \nat} $で,$s_n\to \infty,\; n\to \infty$,ただし$Y(t_n X) = s_nA$,なるものが存在する.

  このとき,$\{\abs{t_n}\}_{n\in \nat}$が有界$\iff \{e^{t_n X}\cdot o_K\}_{n\in \nat}  $が有界ならば,$G/K\ni e^{X}\cdot o_K\mapsto (Y(X), Z(X))\in (\ha\cap \pe)\oplus (\per{\ha}\cap \pe) $,$X\in \pe$が微分同相であることから$\{s_n\}_{n\in \nat} $も有界である.従って対偶より$\lim_{n\to\infty}s_n\to \infty $ならば$\lim_{n\to \infty}\abs{t_n}\to \infty $である.

  任意の$\per{\ha}\cap\pe $の元はある$Z =
  \begin{pmatrix}
    0 & z & w \\
    \bar{z} & 0 & 0 \\
    \bar{w} & 0 & 0 
  \end{pmatrix} \in \per{\ha}\cap\pe $,$z,w\in \quat \st \abs{z}^2 +\abs{w}^2  = 1 $と$r\in \real$により$rZ$と表せる.$Z\in\pe\setminus \ha $より$\re z \neq \pm 1$であることに注意する ($\re\colon \quat\ni a+bi+cj+dk\mapsto a\in \real$とする).$Z(t_n) = r_n\begin{pmatrix}
    0 & z_n & w_n \\
    \bar{z_n} & 0 & 0 \\
    \bar{w_n} & 0 & 0 
  \end{pmatrix} $とする.

  \Cref{lem:exp-quat}より, 
  \begin{align*}
    e^{s_n A}e^{Z(t_n X)}\cdot o_K &=
    \begin{pmatrix}
      \cosh s_n & \sinh s_n & 0 \\
      \sinh s_n & \cosh s_n & 0 \\
      0 & 0 & 1 
    \end{pmatrix}
              \lbig[\begin{pmatrix}
                1\\ \pm \bar{z} \tanh \abs{t_n}  \\ \pm \bar{w} \tanh \abs{t_n}
              \end{pmatrix}\rbig]\\
    &=  \lbig[ \begin{pmatrix}
      \cosh s_n \pm \bar{z} \tanh \abs{t_n} \sinh s_n \\ \sinh s_n \pm \bar{z} \tanh \abs{t_n} \cosh s_n \\ \pm \bar{w} \tanh \abs{t_n}
    \end{pmatrix}\rbig],
  \end{align*}
  複号は$t_n$の符号$\pm$と同順,である.このとき$\lim_{n\to \infty}\tanh s_n = 1 = \lim_{n\to \infty}\tanh \abs{t_n} $と$\re z \neq \pm 1$に注意すると次を得る.具体的な計算は後述する.
  \begin{align}
    \lim_{n\to \infty}(\sinh s_n \pm \bar{z} \tanh \abs{t_n} \cosh s_n)\inv{(\cosh s_n \pm \bar{z} \tanh \abs{t_n} \sinh s_n) } = 1\label{eq:-1}
  \end{align}
  である.

  したがって,$
  \begin{pmatrix}
    0 \\ 0 
  \end{pmatrix}
  \in \quat\mathbb{H}^2 $から$
  \begin{pmatrix}
    1 \\ 0 
  \end{pmatrix}
  \in \quat\mathbb{H}^2 $へのベクトルと,$
  \begin{pmatrix}
    0 \\ 0 
  \end{pmatrix}
  \in \quat\mathbb{H}^2 $から\\
  $ \begin{pmatrix}
    (\sinh s_n \pm \bar{z} \tanh \abs{t_n} \cosh s_n)\inv{(\cosh s_n \pm \bar{z} \tanh \abs{t_n} \sinh s_n) } \\  \ast 
  \end{pmatrix}\in \quat\mathbb{H}^2 $へのベクトルがなす Euclidean な内積の値を $I_n$とすると,$\lim_{n\to \infty}I_n = 1 $である.

  しかし,$
  \begin{pmatrix}
    0 \\ 0 
  \end{pmatrix}
  \in \quat\mathbb{H}^2 $から$
  \begin{pmatrix}
    1 \\ 0
  \end{pmatrix}
  \in \quat\mathbb{H}^2 $へのベクトルと,$
  \begin{pmatrix}
    0 \\ 0 
  \end{pmatrix}
  \in \quat\mathbb{H}^2 $から$e^{W(t_n X)}\cdot o_K \in \quat \mathbb{H}^2 $へのベクトルがなす Euclidean な内積の値$J_n$は,\\
  $W(t_n X)\in \lbig\{
  \begin{pmatrix}
    0 & z_1 & z_2 \\
    \bar{z_1} & 0 & 0\\
    \bar{z_2} & 0 & 0
  \end{pmatrix}
  \relmiddle| z_1, z_2\in \quat,\ \& \ \re z_1 = 0 \rbig\} $であることと,\Cref{lem:exp-quat} から$\re J_n = 0$となり,$e^{s_n A}e^{Z(t_n X)}\cdot o_K = e^{t_n X}\cdot o_K\implies \lim_{n\to \infty} I_n = \lim_{n\to \infty}J_n = 1 $に矛盾する.
  

  以上より「$X\in \pe\setminus\ha \Rightarrow Y(\real X) $有界」,したがって \Cref{prop:1127-main} を得る.
  
\end{pfwn}


\begin{calcof}{\Cref{prop:1127-main}}
  
  $\lim_{n\to \infty}\abs{(\sinh s_n \pm \bar{z} \tanh \abs{t_n} \cosh s_n)\inv{(\cosh s_n \pm \bar{z} \tanh \abs{t_n} \sinh s_n) } - 1} = 0$を示せば主張が得られる.具体的に計算すると,
  \begin{align*}
    &\lim_{n\to \infty}\abs{(\sinh s_n \pm \bar{z} \tanh \abs{t_n} \cosh s_n)\inv{(\cosh s_n \pm \bar{z} \tanh \abs{t_n} \sinh s_n) } - 1}\\
    % &\qquad\qquad\qquad \text{($\text{(この極限)} = 0$を示せば良い)} \\
    = & \lim_{n\to \infty}\lbig|\frac{(\tanh s_n\pm \bar{z}\tanh \abs{t_n})(1 \pm z \tanh\abs{t_n}\tanh s_n )}{\abs{(1 \pm \bar{z} \tanh\abs{t_n}\tanh s_n )}^2} - 1 \rbig|\\    
    = & \lim_{n\to \infty}\frac{\abs{(\tanh s_n\pm \bar{z}\tanh \abs{t_n})z' - (1 \pm \bar{z} \tanh\abs{t_n}\tanh s_n )z' }}{\abs{(1 \pm \bar{z} \tanh\abs{t_n}\tanh s_n )}^2}, \quad z'\defeq 1 \pm z\tanh\abs{t_n}\tanh s_n \\
    = & \lim_{n\to \infty}\frac{\abs{(1-\tanh s_n)(-1\pm \bar{z}\tanh \abs{t_n} )z'}}{\abs{(1 \pm \bar{z} \tanh\abs{t_n}\tanh s_n )}^2} \\
    = & \lim_{n\to \infty}\frac{\abs{(1-\tanh s_n)(-1\pm \bar{z}\tanh \abs{t_n} )}}{\abs{(1 \pm \bar{z} \tanh\abs{t_n}\tanh s_n )}} 
  \end{align*}
  であり,
  $0 < \min{\abs{1\pm \re z}}\leq  \abs{(1 \pm \bar{z} \tanh\abs{t_n}\tanh s_n )}\leq \sqrt{2^2 + 1^2} = \sqrt{5} $と$\min\{\abs{-1\pm \re z } \}  \leq \abs{-1\pm\bar{z}\tanh \abs{t_n} } \leq \sqrt{5} $であることから,
    \begin{align*}
      0 = \lim_{n\to \infty}(1-\tanh s_n)\frac{\min\{\abs{-1\pm \re z} \}}{\sqrt{5}}  &\leq \lim_{n\to \infty}\frac{\abs{(1-\tanh s_n)(-1\pm \bar{z}\tanh \abs{t_n} )}}{\abs{(1 \pm \bar{z} \tanh\abs{t_n}\tanh s_n )}}\\
      &\leq \lim_{n\to \infty}(1-\tanh s_n)\frac{\sqrt{5}}{\min\{\abs{1\pm \re z} \}} = 0
    \end{align*}
    より,\Cref{eq:-1} が成り立つ.

\end{calcof}


%%% Local Variables:
%%% mode: latex
%%% TeX-master: "okuda-master-thesis"
%%% End:
