\subsection{具体例: 実階数1の古典型単純Lie群}
\begin{prop}\label{prop:classical-rank-one}
  $G = \SO(1,n)$,$ \SU(1, n)$,$\Sp(1,n) $,$H = \SO(1,1) $,$n\geq 2$に対して\Cref{yosou:1121} は正しい.
\end{prop}

$G = \Sp(1,2) $,$\ha = \real \begin{pmatrix}
    0 & 1 & 0 \\
    1 & 0 & 0\\
    0 & 0 & 0
  \end{pmatrix}$の場合にのみ示す.その他の場合も全く同様の議論である.
\begin{prop}\label{prop:1127-main}
  $G = \Sp(1,2) $,$H = \SO(1,1)$,$X\in \pe$に対し,$Y(\real X) $が有界$\iff X\in \pe\setminus \ha  $ or $X = 0$である.
\end{prop}

ただし,$H$は$G$の左上に入っている.すなわち,$\Lie H = \ha = \real A $,$A\defeq \begin{pmatrix}
  0 & 1 & 0 \\
  1 & 0 & 0\\
  0 & 0 & 0
\end{pmatrix}$とする.

\begin{nttdef}
  
  $\quat$を四元数体とする.\bluetext{$\Sp(1,2) $の定義}$\Sp(1,2)/\Sp(1)\times \Sp(2) \simeq \{(z_1, z_2)\mid z_1,z_2\in \quat ,\; \abs{z_1}^2 + \abs{z_2}^2   < 1 \} =: \quat\mathbb{H}^2 $である.なぜならば$\trans{(1,0,0)} $の自然表現$\Sp(1,2)\leftaction \quat^2 $による軌道を考え,第2,第3成分に第1成分の逆数を右からかけた空間が$\quat\mathbb{H}^2$と微分同相であるためであり,$\Sp(1,2)\leftaction \quat^3 $の$\trans{(1,0,0)} $軌道の点$
  \begin{pmatrix}
    z_0 \\ z_1 \\ z_2 
  \end{pmatrix}
  $に対応する$\quat\mathbb{H}^2$の点を$
  \lbig[ \begin{pmatrix}
    z_0 \\ z_1 \\ z_2 
  \end{pmatrix}\rbig] = \lbig[ \begin{pmatrix}
    1 \\ z_1\inv{z_0} \\ z_2\inv{z_0} 
  \end{pmatrix}\rbig] 
  $と書く.
\end{nttdef}

愚直な行列計算により,次が示される.
\begin{lem}\label{lem:exp-quat}
  
  $\forall z,w\in \quat$に対し,$\exp
  \begin{pmatrix}
    0 & z & w  \\
    \bar{z} & 0 & 0\\
    \bar{w} & 0 & 0
  \end{pmatrix}
  =
  \begin{pmatrix}
    \cosh r &  \ast & \ast \\
    \\
    \dfrac{\bar{z}}{r} \sinh r &  \ast & \ast \\
    \\
    \dfrac{\bar{w}}{r}\sinh r &  \ast & \ast 
  \end{pmatrix}
  $,ただし$r \defeq \sqrt{\abs{z}^2 + \abs{w}^2 } $,である.
\end{lem}

\begin{npfwn}[\Cref{prop:1127-main}]
  
  % $\Lie H = \ha = \real A $,$A\defeq \begin{pmatrix}
  %   0 & 1 & 0 \\
  %   1 & 0 & 0\\
  %   0 & 0 & 0
  % \end{pmatrix}$とする.
  $X = 0\Rightarrow Y(\real X) = \{0\} $と$X\in \ha\setminus\{0\} $のときに$Y(\real X) $が非有界であることは明らかであるから,$X\nin \ha $の場合にのみ議論すればよい.% つまり
  % $X =
  % \begin{pmatrix}
  %   0 & z & w \\
  %   \bar{z} & 0 & 0 \\
  %   \bar{w} & 0 & 0 
  % \end{pmatrix}
  % \in \pe\setminus\ha $,$z,w\in \quat \st \abs{z}^2 +\abs{w}^2  = 1 $を任意に1つ固定して議論して一般性を失わない.このとき,$X\in\pe\setminus \ha $より$\re z \neq \pm 1$であることに注意する ($\re\colon \quat\ni a+bi+cj+dk\mapsto a\in \real$とする).

  $G$の Cartan 対合を$\Theta(g) = \inv{(g^{*})} $ ($g^{*}$は$g$の共役転置) とするとき,$\Theta(e^{Y(tX)}e^{Z(tX)})\cdot o_K = e^{-Y(tX)}e^{-Z(tX)}\cdot o_K = \Theta(e^{X})\cdot o_K = e^{-X}\cdot o_K $より,「$Y(\real X) $が非有界$\iff Y(\real X)\subset \real A $が上に非有界」である.

  したがって,$Y (\real X) $が非有界であるとき,列$\{t_n \in \real \}_{n\in \nat} $で,$s_n\to \infty,\; n\to \infty$,ただし$Y(t_n X) = s_nA$,なるものが存在する.

  また,任意の$\per{\ha}\cap\pe $の元はある$Z =
  \begin{pmatrix}
    0 & z & w \\
    \bar{z} & 0 & 0 \\
    \bar{w} & 0 & 0 
  \end{pmatrix} \in \per{\ha}\cap\pe $,$z,w\in \quat \st \abs{z}^2 +\abs{w}^2  = 1 $と$r\in \real$により$rZ$と表せる.$Z(t_nX) = r_nZ_n$,$Z_n\defeq \begin{pmatrix}
    0 & z_n & w_n \\
    \bar{z_n} & 0 & 0 \\
    \bar{w_n} & 0 & 0 
  \end{pmatrix} $,$z_n,w_n\in \quat \st \abs{z_n}^2 +\abs{w_n}^2  = 1 $とすると,$X\nin\ha$であるから\Cref{thm:kob97}より$\abs{r_n}\to \infty $である.$z_n,w_n\in \quat \st \abs{z_n}^2 +\abs{w_n}^2  = 1 $より,$\{t_n\} $の部分列を取るとある$Z_{\infty} $が存在して$\lim_{n\to \infty}Z_n = Z_{\infty} =
  \begin{pmatrix}
    0 & z_{\infty} & w_{\infty} \\
    \bar{z_{\infty}} & 0 & 0 \\
    \bar{w_{\infty}} & 0 & 0 
  \end{pmatrix}
  \in \per{\ha}\cap\pe $なるようにできる.$Z\in\pe\setminus \ha $より$\re z_{\infty} \neq \pm 1$であることに注意する ($\re\colon \quat\ni a+bi+cj+dk\mapsto a\in \real$とする).

  \Cref{lem:exp-quat}より, 
  \begin{align*}
    e^{s_n A}e^{r_n Z_n}\cdot o_K &=
    \begin{pmatrix}
      \cosh s_n & \sinh s_n & 0 \\
      \sinh s_n & \cosh s_n & 0 \\
      0 & 0 & 1 
    \end{pmatrix}
              \lbig[\begin{pmatrix}
                1\\ \pm \bar{z_n} \tanh \abs{r_n}  \\ \pm \bar{w_n} \tanh \abs{r_n}
              \end{pmatrix}\rbig]\\
    &=  \lbig[ \begin{pmatrix}
      \cosh s_n \pm \bar{z_n} \tanh \abs{r_n} \sinh s_n \\ \sinh s_n \pm \bar{z_n} \tanh \abs{r_n} \cosh s_n \\ \pm \bar{w_n} \tanh \abs{r_n}
    \end{pmatrix}\rbig],
  \end{align*}
  複号は$r_n$の符号$\pm$と同順,である.このとき$\lim_{n\to \infty}\tanh s_n = 1 = \lim_{n\to \infty}\tanh \abs{r_n} $と$\lim_{n\to \infty} \re z_n = \re z_{\infty} \neq \pm 1$に注意すると次を得る.具体的な計算は後述する.
  \begin{align}
    \lim_{n\to \infty}(\sinh s_n \pm \bar{z_n} \tanh \abs{r_n} \cosh s_n)\inv{(\cosh s_n \pm \bar{z_n} \tanh \abs{r_n} \sinh s_n) } = 1\label{eq:-1}
  \end{align}
  である.

  したがって,$
  \begin{pmatrix}
    0 \\ 0 
  \end{pmatrix}
  \in \quat\mathbb{H}^2 $から$
  \begin{pmatrix}
    1 \\ 0 
  \end{pmatrix}
  \in \quat\mathbb{H}^2 $へのベクトルと,$
  \begin{pmatrix}
    0 \\ 0 
  \end{pmatrix}
  \in \quat\mathbb{H}^2 $から\\
  $ \begin{pmatrix}
    (\sinh s_n \pm \bar{z_n} \tanh \abs{r_n} \cosh s_n)\inv{(\cosh s_n \pm \bar{z_n} \tanh \abs{r_n} \sinh s_n) } \\  \ast 
  \end{pmatrix}\in \quat\mathbb{H}^2 $へのベクトルがなすEuclideanな内積の値を $I_n$とすると,$\lim_{n\to \infty}I_n = 1 $である.

  しかし,$
  \begin{pmatrix}
    0 \\ 0 
  \end{pmatrix}
  \in \quat\mathbb{H}^2 $から$
  \begin{pmatrix}
    1 \\ 0
  \end{pmatrix}
  \in \quat\mathbb{H}^2 $へのベクトルと,$
  \begin{pmatrix}
    0 \\ 0 
  \end{pmatrix}
  \in \quat\mathbb{H}^2 $から$e^{t_nX}\cdot o_K \in \quat \mathbb{H}^2 $へのベクトルがなすEuclideanな内積の値$J_n$は,$X \defeq   \begin{pmatrix}
    0 & z_0 & w_0 \\
    \bar{z_0} & 0 & 0 \\
    \bar{w_0} & 0 & 0 
  \end{pmatrix}$,$z_0,w_0\in \quat \st \abs{z_0}^2 +\abs{w_0}^2  = 1 $とするとき$J_n = \dfrac{\bar{z_0}}{r_0}\tanh (tr_0) $,$r_0\defeq \sqrt{\abs{z_0}^2 + \abs{w_0}^2 } $であり,$X\nin ha \iff z_0\neq 1$より$\lim_{n\to \infty}J_n = \dfrac{\bar{z_0}}{r_0}\neq 1 $である.

  以上2つの議論を合わせると$e^{s_n A}e^{r_n Z_n}\cdot o_K = e^{t_n X}\cdot o_K\implies 1 = \lim_{n\to \infty} I_n = \lim_{n\to \infty}J_n \neq 1 $となり矛盾する.
  

  以上より「$X\in \pe\setminus\ha \Rightarrow Y(\real X) $有界」,したがって \Cref{prop:1127-main} を得る.
  
\end{npfwn}


\begin{ncalcof}[\Cref{prop:1127-main}]
  
  $\lim_{n\to \infty}\abs{(\sinh s_n \pm \bar{z_n} \tanh \abs{r_n} \cosh s_n)\inv{(\cosh s_n \pm \bar{z_n} \tanh \abs{r_n} \sinh s_n) } - 1} = 0$を示せば主張が得られる.具体的に計算すると,
  \begin{align}
    &\lim_{n\to \infty}\abs{(\sinh s_n \pm \bar{z_n} \tanh \abs{r_n} \cosh s_n)\inv{(\cosh s_n \pm \bar{z_n} \tanh \abs{r_n} \sinh s_n) } - 1}\notag \\
    % &\qquad\qquad\qquad \text{($\text{(この極限)} = 0$を示せば良い)} \\
    = & \lim_{n\to \infty}\lbig|\frac{(\tanh s_n\pm \bar{z_n}\tanh \abs{r_n})(1 \pm z_n \tanh\abs{r_n}\tanh s_n )}{\abs{(1 \pm \bar{z_n} \tanh\abs{r_n}\tanh s_n )}^2} - 1 \rbig|\notag \\    
    = & \lim_{n\to \infty}\frac{\abs{(\tanh s_n\pm \bar{z_n}\tanh \abs{r_n})z_n' - (1 \pm \bar{z_n} \tanh\abs{r_n}\tanh s_n )z_n' }}{\abs{(1 \pm \bar{z_n} \tanh\abs{r_n}\tanh s_n )}^2}\label{eq:ast} \tag{$\ast$}
  \end{align}
  である.ここで$ z_n'\defeq 1 \pm z_n\tanh\abs{r_n}\tanh s_n$とすると,
  \begin{align*}
    (\ast) &= \lim_{n\to \infty}\frac{\abs{(\tanh s_n\pm \bar{z_n}\tanh \abs{r_n})z_n' - (1 \pm \bar{z_n} \tanh\abs{r_n}\tanh s_n )z_n' }}{\abs{(1 \pm \bar{z_n} \tanh\abs{r_n}\tanh s_n )}^2}\\
    = & \lim_{n\to \infty}\frac{\abs{(1-\tanh s_n)(-1\pm \bar{z_n}\tanh \abs{r_n} )z_n'}}{\abs{(1 \pm \bar{z_n} \tanh\abs{r_n}\tanh s_n )}^2} \\
    = & \lim_{n\to \infty}\frac{\abs{(1-\tanh s_n)(-1\pm \bar{z_n}\tanh \abs{r_n} )}}{\abs{(1 \pm \bar{z_n} \tanh\abs{r_n}\tanh s_n )}} 
  \end{align*}
  であり,
  $0 < \min{\abs{1\pm \re z_n}}\leq  \abs{(1 \pm \bar{z_n} \tanh\abs{r_n}\tanh s_n )}\leq \sqrt{2^2 + 1^2} = \sqrt{5} $と$\min\{\abs{-1\pm \re z_n } \}  \leq \abs{-1\pm\bar{z_n}\tanh \abs{r_n} } \leq \sqrt{5} $であることと$\lim_{n\to \infty} \re z_n = \re z_{\infty} \neq \pm 1$より,
    \begin{align*}
      0 = \lim_{n\to \infty}(1-\tanh s_n)\frac{\min\{\abs{-1\pm \re z_n} \}}{\sqrt{5}}  &\leq \lim_{n\to \infty}\frac{\abs{(1-\tanh s_n)(-1\pm \bar{z_n}\tanh \abs{r_n} )}}{\abs{(1 \pm \bar{z_n} \tanh\abs{r_n}\tanh s_n )}}\\
      &\leq \lim_{n\to \infty}(1-\tanh s_n)\frac{\sqrt{5}}{\min\{\abs{1\pm \re z_n} \}} = 0
    \end{align*}
    より,\Cref{eq:-1} が成り立つ.
\end{ncalcof}


%%% Local Variables:
%%% mode: latex
%%% TeX-master: "okuda-master-thesis"
%%% End:
