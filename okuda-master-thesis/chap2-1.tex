\subsection{$ G$の実階数が1の場合}

\begin{thm}\label{thm:1216-main}
  $G$を実階数1の実半単純Lie 群とするとき,\Cref{yosou:1121} が成り立つ.
\end{thm}

\bluetext{$\SU(1,2) $-reduction については2021/8/10のレポートを参考にせよ}

\subsubsection{補足: \Cref{thm:1216-main}の微分幾何的側面}
\begin{defi}{\cite[Definition~1.3]{e72-1}}\label{def:visibility}

  $M$が完備かつ非正曲率をもつ1-連結Riemann多様体であるとき,$M$をHadamard多様体といい,Hadamard多様体$M$が visibility manifold であるとは,$\forall p\in M, \forall \epsilon > 0$に対し,ある$r(p,\epsilon) >0 $が存在して,測地線$\gamma\colon [t_0, t_1]\to X $が$d_{M}(p, \gamma(t))\geq r(p,\epsilon) $,$\forall t\in [t_0, t_1]$ならば,$\measuredangle_{p}(\gamma(t_0), \gamma(t_1)) \leq \epsilon $であることである.
\end{defi}

\bluetext{図を入れる}

\begin{thm}{\cite[p.~296, 9.33~Theorem]{bh99}, originally \cite[Theorem~4.1]{e72-2}}\label{thm:visibility-and-rank}
  
  $\exists C\cptsub M\st M = \bigcup\{f(C)\mid f\in \isom(M) \}  $なるHadamard多様体$M$に対し,次は同値である.
  \vspace{-1em}
  \begin{enumerate}
    \renewcommand{\labelenumi}{(\roman{enumi})}
  \item $M$はvisibility manifoldである.
  \item 全測地的な部分Riemann多様体$M'\subset M$で$\real^2$と等長同型なものが存在しない.
  \end{enumerate}
\end{thm}

ここでRiemann対称空間はHadamard多様体であり,\Cref{thm:visibility-and-rank}の (ii) は$G$の実階数が1以下であることと同値である.したがって$G$の実階数が1の場合$G/K$はvisibility manifoldであり,$G = \SU(1,2) $,$H= \SO(1,1)$の場合の証明と全く同様にして背理法により\Cref{yosou:1121}が示される.
