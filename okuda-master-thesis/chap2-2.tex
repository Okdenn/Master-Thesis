\subsection{$G$が実階数1の実半単純Lie群の直積の場合}
\Cref{thm:1216-main}の系として次が示される.
\begin{cor}\label{cor:0113-main}
  $n\in \nat$を固定し,$\{G_i \}_{1\leq i\leq n} $を実階数1の実半単純Lie群の族,$\Theta_i $を$G_i$のCartan対合とする.
  
  $G$を$\{G_i \}_{1\leq i\leq n} $の直積からなるLie群$G = G_1\times \cdots \times G_n $とし,$H$を$G$の非コンパクトな部分Lie群で,$G$のCartan対合$\Theta\colon G\to G $,$G\ni (g_1\cldots g_n)\mapsto (\Theta_1g_1\cldots \Theta_ng_n )\in G $に対して$\Theta H = H$で$\dim \ha = 1$なるものとする.このとき\Cref{yosou:1121}が成り立つ.
\end{cor}


\begin{pfwn}{\Cref{cor:0113-main}}

  各$G_i $を$G$の部分Lie群と自然にみなす.このとき$H_i\defeq G_i\cap H$とすると,$H \simeq H_1\times\cdots  \times H_n  $である.同様に$\ge_i\subset \ge = \bigoplus_{1\leq i\leq n}\ge_i $とみなすと$\ha_i\defeq \ge_i\cap \ha $は$H_i$のLie環である.

  $K_i$を$\Theta K_i = K_i $なる$G_i$の極大コンパクト部分群とすると,$K \defeq K_1\times \cdots \times K_n $は$G$の極大コンパクト部分群で$\Theta K = K $を満たす.

  $G/K\simeq G_1/K_1\times \cdots \times G_n/K_n $であり,\Cref{thm:kob89-lem6.1}により各$1\leq i\leq n$の$(G_i, H_i, \Theta_i) $に対し上への微分同相$\pi_i\colon  (\ha_i\cap\pe_i)\oplus (\per{\ha_i}\cap \pe_i) \ni (Y, Z)\mapsto e^{Y}e^{Z}\cdot o_K \in G_i/K_i $が存在する.$X_i\in \pe_i $に対し$(Y_i(X_i), Z_i(X_i))\defeq \inv{\pi_i}(e^{X_i}K_i) $と定める.

  $X\in \pe$に対し,$X = X_1 +\cdots + X_n $を$\pe = \bigoplus_{1\leq i\leq n}\pe_i $に対応する$X$の分解とすると,$Y(\real X) $が有界であることは各$Y_i(\real X_i) $が有界であることと同値である.また\Cref{thm:1216-main}より$Y_i(\real X_i) $が有界であることと「$X_i = 0 $あるいは$X_i\in \pe\setminus\ha $であること」が同値である.この2つを合わせると,

  
  
\end{pfwn}