\subsection{$G$が実階数1の群の直積の場合}

\begin{thm}\label{thm:0113-main}
  $n\in \nat$を固定し,$\{G_i \}_{1\leq i\leq n} $を実階数1の実半単純Lie群の族とする.
  $G$を$\{G_i \}_{1\leq i\leq n} $の直積からなるLie群$G = G_1\times \cdots \times G_n $とし,$H$を$G$の非コンパクトな部分Lie群で,$G$のCartan対合$\Theta$に対して$\Theta H = H$で$\dim \ha = 1$なるものとする.このとき\Cref{yosou:1121}が成り立つ.
\end{thm}


\begin{pfwn}{\Cref{thm:0113-main}}

  各$G_i $を$G$の部分Lie群と自然にみなす.このとき$H_i\defeq G_i\cap H$とすると,$H \simeq H_1\times\cdots  \times H_n  $である.同様に$\ge_i\subset \ge = \bigoplus_{1\leq i\leq n}\ge_i $とみなすと$\ha_i\defeq \ge_i\cap \ha $は$H_i$のLie環である.
  
\end{pfwn}