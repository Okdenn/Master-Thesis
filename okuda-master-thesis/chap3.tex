$G$の実階数が2のとき\Cref{yosou:1121}には反例が存在する.

\begin{prop}\label{prop:0114}
  $G = \SL(3,\real) $,$H = \{\diag(e^a,e^b,e^c)\mid a,b,c\in \real,\ a+ b + c = 1 \} $,$X\defeq
  \begin{pmatrix}
    1 & 0 & 0 \\
    0 & 0 & \sqrt{2} \\
    0 & \sqrt{2} & -1
  \end{pmatrix}
  $に対し$Y(\real X) $は非有界である.
\end{prop}


$\ha = \{\diag(a,b,c)\mid a,b,c\in \real,\ a +b + c = 0 \}  $であるから$X_1 = \diag(1,0,-1)$,$X_2 = \begin{pmatrix}
  0 & 0 & 0 \\
  0 & 0 & \sqrt{2} \\
  0 & \sqrt{2} & 0
\end{pmatrix}$であり,$[X_1, X_2] = \begin{pmatrix}
  0 & 0 & 0 \\
  0 & 0 & \sqrt{2} \\
  0 & -\sqrt{2} & 0
\end{pmatrix} \neq 0$より$X$は\Cref{yosou:1121}の右辺の集合には入っているが,$X\nin \pe_{H,\bdd} $であるから,\Cref{prop:0114}は\Cref{yosou:1121}の反例となっている.

\begin{npfwn}[\Cref{prop:0114}]
  行列式1の$3\times 3$正定値実対称行列全体の集合$\Symm^{+}(3)$と$G/K$は$gK \mapsto gI_2\trans{g} $により微分同相である.
  
  $e^{2tX} $を計算するための補題を用意する.
  
  \begin{lem}
    $\cos^2 \theta = \frac{2}{3} $を満たす$\theta$に対し,$k =
    \begin{pmatrix}
      \cos \theta & -\sin \theta \\ \sin \theta & \cos \theta
    \end{pmatrix}
    $とすると,$t\in \real$に対し
    \begin{align*}
      k\exp\lbig(2t\begin{pmatrix}
        0 & \sqrt{2} \\
        \sqrt{2} & -1 
      \end{pmatrix}\rbig)\inv{k} =
                   \begin{pmatrix}
                     \dfrac{2e^{2t} + }
                   \end{pmatrix}
    \end{align*}
    である.
  \end{lem}
  
\end{npfwn}
