$G$を非コンパクトな実半単純Lie群,$H$を$G$の部分Lie群で,$G$のCartan対合$\Theta$に対して$H = \Theta H$を満たすものとする.$K$を$G$の極大コンパクト部分群で$K = \Theta K $なるものとするとき,$G/K$は$\ge$のKilling形式$B$から定まるRiemann計量によってRiemann多様体の構造を持つ.$\ge = \ka \oplus \pe $を$\Theta$の微分$d\Theta$による$\ge$のCartan分解とするとき,$G/K$は$\pe$と微分同相であり,$eK$を通る$G/K$の極大測地線は$B(X, X) = 1 $なる$X\in \pe$によって$e^{tX}K $,$t\in \real$と書ける.測地線$e^{tX}K$の$\ha\cap\pe$成分と$\per{\ha}\cap\pe$成分への分解を与える定理として次の定理が知られている.

\begin{thm*}\cite[Lemma~6.1]{kob89}\label{thm:kob89-lem6.1}

  $\pi\colon  (\ha\cap\pe)\oplus (\per{\ha}\cap \pe) \ni (Y, Z)\mapsto e^{Y}e^{Z}\cdot o_K \in G/K $は上への微分同相である.
\end{thm*}
この定理を用いると$X\in \pe$に対し,$(Y(X), Z(X))\defeq \inv{\pi}(e^X\cdot o_K)\in (\ha\cap\pe)\oplus (\per{\ha}\cap \pe)$と定義できる.
