% \documentclass[uplatex]{jsreport}
% \usepackage{amssymb,amsthm,amsmath}
% \theoremstyle{definition}
% \renewcommand{\proofname}{\upshape \bfseries 証明}
% \renewcommand{\labelenumi}{(\roman{enumi})}

% \allowdisplaybreaks[1]


\RequirePackage{plautopatch}
\plautopatchdisable{eso-pic} % https://oku.edu.mie-u.ac.jp/tex/mod/forum/discuss.php?d=2956, pdfpages との相性を良くする
\documentclass[12pt,dvipdfmx,uplatex]{jsarticle}
% \documentclass[11pt,autodetect-engine,dvi=dvipdfmx,ja=standard,
% label-section=modern, % モダーン!(Theorem Appendix A.3 みたいなアホがない)
% a4paper]{bxjsarticle}

\input{../../../preamble-j}
% \usepackage[margin=15mm]{geometry}
% \usepackage{geometry}
\usepackage{multirow}
\input{../../../pikachu}
\input{../../../colorsforprint.tex}
\usepackage{enumitem}
\usepackage{pdfpages} % at the final compile, delete the option "demo"
\usepackage[top=10truemm,bottom=15truemm,left=12truemm,right=12truemm]{geometry} %DO NOT option ``a4paper", it changes the size of outputed PDF
% \newtheorem{thm}{定理}[section]
% \newtheorem{lem}[thm]{補題}
% \newtheorem{defi}[thm]{定義}
% \newtheorem{cor}[thm]{系}
% \newtheorem{exam}[thm]{例}
% \newtheorem{rem}[thm]{注意}
% \newtheorem{conj}[thm]{予想}
% \newtheorem{ques}[thm]{問題}


\Crefname{figure}{図}{図}
\crefname{figure}{図}{図}

% \AtBeginEnvironment{thm}{\begin{minipage}{\textwidth}\vspace{1em}}
% \AtEndEnvironment{thm}{\end{minipage}}


\DeclareMathOperator{\ballsub}{\underset{\text{ball}}{\subset}}
\renewcommand{\dx}{\; dx}
\newcommand{\uah}{\underline{\ah}}
\newcommand{\MJ}{MJ}
\newcommand{\img}{\sqrt{-1}}
% \DeclareMathOperator{\MJ}{MJ}

\newcommand{\bigzero}{\mbox{\Large\textbf{0}}}
\newcommand{\rvline}{\hspace*{-\arraycolsep}\vline\hspace*{-\arraycolsep}}
\newcommand\Tstrut{\rule{0pt}{2.6ex}}         % = `top' strut
\newcommand\Bstrut{\rule[-0.9ex]{0pt}{0pt}}   % = `bottom' strut

\makeatletter
\newcommand{\neutralize}[1]{\expandafter\let\csname c@#1\endcsname\count@}
\makeatother

\newenvironment{claimbis}[1]
  {\renewcommand{\thethm}{\ref{#1}$'$}%
   \neutralize{claim}\phantomsection
   \begin{claim}}
  {\end{claim}}

% \begin{align*}
%   \begin{array}{ccc}
%     M & \stackrel{\phi}{\longrightarrow} & M' \\
%     \rotatebox{90}{$\in$} & & \rotatebox{90}{$\in$} \\
%     p & \longmapsto & \phi(p)
%   \end{array}
% \end{align*}

% \begin{figure}[H]
%   \centering
%   \begin{tikzpicture}
%     \node (ef) at (0, 1.2) {$E\times F$};
%     \node (g) at (1.2, 1.2) {$G$};
%     \node (m) at (0, 0) {$M$};
%     \draw[->] (ef) to node[above] {$\scriptstyle b$} (g);
%     \draw[->] (ef) to node[left] {$\scriptstyle \phi$} (m);
%     \draw[->] (m) to node[right] {$\scriptstyle \exists !\tilde{b}$} (g);
%     \node (a) at (0.4, 0.7) {$\scriptstyle \circlearrowleft$};
%   \end{tikzpicture}
% \end{figure}

% \begin{empheq}[left={|x|=\empheqlbrace}]{alignat*=2}
%   x & \quad (x\geq 0) \\
%   -x & \quad (otherwise) 
% \end{empheq}

% easylist
% \ListProperties(Margin2=2em,Space1*=0em,FinalMark={)}, Style2*=\textbullet, Hide2=2)

% inline eqs. w/. numbering
% e.g.
% \inlineequation[eq:matsushima-p191-1]{\forall X\in\ge,\ \ad(X)\circ J = J\circ \ad(X)}

% \begin{figure}[H]
%   \centering
%   %   \raggedleft
%   %   \raggedrightp
%   %   \includegraphics[scale=0.08]{graph/fig1.jpg}
%   \includegraphics[scale=0.08]{graph/fig1-2.jpg}
%   \label{fig:fig1}
% \end{figure}


% % \maketitleの余白を調整
% \makeatletter
% \renewcommand{\@maketitle}{\newpage
%   % \null
%   % \vskip 2em 
%   \begin{center}
%     {\LARGE \@title \par} % \vskip 1.5em
%     {\large \lineskip .5em
%       \begin{tabular}[t]{c}\@author
%       \end{tabular}\par
%     }
%     % \vskip 1em
%     % {\large \@date}
%   \end{center}
%   % \par
%   % \vskip 0.5em
% }
% \makeatother

\usepackage{advdate} % \AdvanceDate[-1]\today の進捗報告,みたいな



\makeatletter
\@addtoreset{equation}{section}
\def\theequation{\thesection.\arabic{equation}}
\makeatother

\begin{document}

\begin{titlepage}
% \newgeometry{top=10truemm,bottom=15truemm,left=12truemm,right=12truemm}
\huge
\centering
{\Huge 2021年度}

\vspace{4cm}

% \raggedright
\begin{center}
  修士論文題目
\end{center}

% \flushleft
\centering
\noindent
\underline{\large Riemann 対称空間上における測地線の簡約部分 Lie 代数への射影に対する有界性}

\underline{\large ---低階数・低次元の場合---}

\vspace{10cm}

\Large
\raggedright
\begin{tabular}{ll}
学生証番号 \quad {} & \underline{45-196010} \\
フリガナ & オクダ タカコ \\
氏名 & \underline{奥田 堯子}
\end{tabular}

\end{titlepage}

\newgeometry{}

\makeatletter

\renewcommand*{\l@section}[2]{%
  \ifnum \c@tocdepth >\z@
    \addpenalty{\@secpenalty}%
    \addvspace{1.0em \@plus\p@}%
    \begingroup
      \parindent\z@
      \rightskip\@tocrmarg
      \parfillskip-\rightskip
      \leavevmode\headfont
      \setlength\@lnumwidth{1.5em}% 元1.5em [2003-03-02]
      \advance\leftskip\@lnumwidth \hskip-\leftskip
      #1\nobreak
      \leaders\hbox{\normalfont$\m@th \mkern \@dotsep mu\hbox{.}\mkern \@dotsep mu$}\hfill %
      \nobreak\hbox to\@pnumwidth{\hss#2}\par
    \endgroup
  \fi}

\makeatother

% {% \hypersetup{linkbordercolor=black}
%   \hypersetup{linkcolor=black}
% % or \hypersetup{linkcolor=black}, if the colorlinks=true option of hyperref is used
% \tableofcontents
% }

\setcounter{tocdepth}{3}
\tableofcontents

\allowdisplaybreaks[0]

\clearpage

\section*{導入}
\addcontentsline{toc}{section}{導入}
$G$を非コンパクトな実線型半単純Lie群,$K$を$G$の極大コンパクト部分群で$G$のCartan対合$\Theta$に対して$K = \Theta K $なるものとする.$\ge = \ka \oplus \pe $を$\Theta$の微分$d\Theta$による$\ge$のCartan分解とするとき,$G/K$は$\pe$と$\pe \ni X\mapsto e^{X}K\in G/K $により微分同相である.

$H$を$G$の非コンパクトかつ連結成分有限個の閉部分群で,$H = \Theta H$を満たすものとし,$\ge$のKilling形式を$B$とする.$\per{\ha}\defeq \{W\in \ge\mid B(W, \ha) = \{0\} \} $とするとき,$G/K$と$\pe$の微分同相についてより強い次の構造定理が知られている.

\begin{thm*}(\cite[Lemma~6.1]{kob89})\label{thm:kob89-lem6.1}  
  $\pi\colon  (\ha\cap\pe)\oplus (\per{\ha}\cap \pe) \ni (Y, Z)\mapsto e^{Y}e^{Z}\cdot o_K \in G/K $は上への微分同相である.
\end{thm*}
この定理を用いて$X\in \pe$に対し,${(Y(X), Z(X))\defeq \inv{\pi}(e^X\cdot K)\in (\ha\cap\pe)\oplus (\per{\ha}\cap \pe)}$と定義する.

$G/K$に$\ge$のKilling形式$B$から定まるRiemann計量によってRiemann多様体の構造を定める.$G$の単位元の$G/K$での像$eK$を通る$G/K$上の任意の極大測地線は$B(X, X) = 1 $なる$X\in \pe$と$t\in \real$によって$e^{tX}K $と書ける.\Cref{thm:kob89-lem6.1}より任意の$t\in \real$に対して$e^{tX}K = e^{Y(tX)}e^{Z(tX)}K $である.

$t\in \real$に対し$e^{Y(tX)}K $は$e^{tX}K $から$eK$の$H$軌道に下ろした垂線の足であるという幾何学的な捉え方ができる.\Cref{fig:y-and-z}は具体的に$G = \SU(1,1) $,$H = \SO(1,1) $としたとき,{\Poincare}円板$G/K$における$e^{Y(X)}K $などの位置関係を示したものであり,\Cref{fig:y-and-z}において測地線$e^{tX}K$ (赤色の斜め線) とその上の一点$e^{tX}K$から$eK$の$H$軌道 (中央の直線) に下ろした垂線の足 (緑の丸) が$e^{Y(tX)}K $である.



\begin{figure}[H]
  \centering
  %   \raggedleft
  %   \raggedrightp
  %   \includegraphics[scale=0.08]{graph/fig1.jpg}
  \includegraphics[scale=0.3]{../graph/y-and-z.pdf}
  \caption{{\Poincare}円板における$Y(tX) $の幾何学的意味}
  \label{fig:y-and-z}
\end{figure}

本論文では小林俊行氏による次の問題 (後述の\Cref{prob:1121}) について考察し,$G$が実階数1の実半単純Lie群の場合に肯定的な結果を得た.
\begin{prob*}(小林俊行氏による)
  $X\in  \pe$に対して,$Y(\real X)$が$ \ha\cap \pe$の有界な部分集合であることと次の条件は同値であるか?
  \begin{cond*}
    \leavevmode
    % \vspace{-1em}
    \begin{itemize}
    \item $X\in \per{\ha}\cap\pe $
    \item[] もしくは
    \item $[X_1, X_2] \neq 0 $かつ$X\in \ge_{s}' $,
    \item[] もしくは
    \item $[X_1, X_2] \neq 0 $かつ$\pe\cap \ze(\ge') \nsubset \ha  $
    \end{itemize}
    である.
  \end{cond*}

  記号は次のとおりとする.
  \begin{itemize}
  \item $X = X_1 + X_2 $はベクトル空間としての分解$\pe =(\pe\cap \ha)\oplus(\pe\cap\per{\ha}) $に対応する$X\in \pe$の分解とする.
  \item $\ge ' $は$\ha$と$X$が生成する$\ge$の部分Lie環とする.$\ha = \theta \ha$より$\theta \ge' = \ge'$であるから$\ge'_{s} \defeq [\ge',\ge'] $,$\ze(\ge') $を$\ge'$の中心とすると$\ge' = \ze(\ge') \oplus \ge'_{s} $である.
  \end{itemize}
\end{prob*}

ここで$G$が実階数1のとき,上の条件と$X\in \{0\}\cup\pe\setminus\ha $は同値である (後述する\Cref{lem:basic-prob}の3).



\Cref{prob:1121}の背景を説明するために\cite{ber88}の内容についてふれる.

まずいくつか用語と命題を準備する.以下では$G$を実簡約Lie群,$H$を$G$の閉部分群とし,$G/H$には左Haar測度$\mu_{G/H} $が存在すると仮定する.
\begin{defi*}(\cite{ber88})
  \leavevmode\vspace{-1em}
  \begin{itemize}
  \item 局所有界関数$r\colon G\to \real_{\geq 0} $がproperなradial functionであるとは,$r$が次の4条件を満たすことである.
    \begin{enumerate}
    \item $e\in G$を単位元とするとき$r(e) = 0 $である.
    \item 任意の$g\in G$に対し$r(g) = r(\inv{g})\geq 0  $である.
    \item 任意の$g_1,g_2\in G$に対し$r(g_1g_2)\leq r(g_1) + r(g_2)  $である.
    \item 任意の$R\geq 0$に対し,$B(R)\defeq \{g\in G\mid r(g)\leq R \} $は$G$の相対コンパクト集合である.
    \end{enumerate}
  \item properなradial function $r\colon G\to \real_{\geq 0} $から$r_{G/H}(gH)\defeq \inf_{h\in H}\{r(gh) \}$により定まる${r_{G/H}\colon G/H\to \real_{\geq 0}}$を$G/H$上のradial functionという.
  \item $d = \inf\{d'\geq 0\mid \text{ある } C > 0\text{が存在して }  m_X(B(r))\leq C(1+r)^{d'}\} $であるとき,$G/H$のランクは$d$であると言う.
  \item $G$の連続表現$\cal{V} $のsmooth vector全体の集合を$\cal{V}^{\infty} $とする.
  \end{itemize}
\end{defi*}
\begin{thmdef*}(\cite[p.~683]{ber88})
  $G/H$には次を満たす非自明なBorel 測度$m_X $ (standard measure) が定数倍を除いて一意的に存在する.

  単位元のコンパクトな近傍で$B = \inv{B} $なる任意の$B\subset G$と任意の$g\in B$,$x\in G/H$に対し,ある定数$C_B\geq 0 $が存在して$g\cdot m_X \leq C_B m_X$,$ \inv{C_B} < m_X(Bx) < C_{B}$である.
\end{thmdef*}
\begin{thmdef*}(\cite[p.~678]{ber88})
  $G/H$の左Haar測度$\mu_{G/H} $を1つ固定する.次の同型写像が存在する.$\Hom_{G}((C_c(G/H))^{\infty}, V )\simarrow \Hom_{G}(V^{\infty}, C(G/H)^{\infty}) $,$\alpha_V\mapsto \beta_V$ただし任意の$v\in V$,$\phi \in  (C_c(G/H))^{\infty} $に対し$ \lyama v,\alpha_V(\phi)\ryama_{V} = \dint_{G/H}\beta_V(v)\phi d\mu_X  $である.
\end{thmdef*}


\begin{thm*}(\cite[pp.~665--6]{ber88})\label{thm:plancherel}
  $G/H$のランクが$d$であるとき,$G$の既約ユニタリ表現$V$が$G$の正則表現$L^2(G/H)$の既約分解に出現する必要条件は,非自明な$G$絡作用素$\alpha_V\colon (C_c(G/H))^{\infty}\to V $が存在し,かつ任意の$v\in V^{\infty} $,$d' > d$に対して
  \begin{align}
    \dint_{G/H}\lbig|\beta_V(v)(x)(1+r(x))^{-d/2} \rbig|^2\; dx < \infty\label{eq:ber88}\tag{$\bigstar$}
  \end{align}
  なることである.% ただし$\beta_V $は次の命題により$\alpha_V $と対応する$G$絡作用素$\beta_V\colon V^{\infty}\to C(G/H)^{\infty}  $とする.
\end{thm*}

$L^2(G/H)$の既約分解に現れる表現を\Cref{eq:ber88}を用いて分析するためには$G/H$のランクを知る必要がある.

% $G,H$が対称対であるときを考える.
$H$を実線型簡約Lie群$G$の非コンパクトかつ連結成分有限個の閉部分群で,$G$のCartan対合$\Theta$に対して$\Theta H = H$なるものとする.

今さらに$\Theta $と可換な$G$の対合$\sigma$に対して$G_{\sigma}\defeq \{g\in G \mid \sigma g = g \} $とし,$H$は$G_{\sigma} $の開部分群であるとし,$\ge = \ha\oplus \qu $を$d\sigma $は$\ha$上1,$\qu$上${-1}$であるような$d\sigma$の固有空間分解とする.このとき極大可換部分空間$\ah\subset \pe\cap \qu $と正ルートの集合$\Sigma^{+}() $


$Y(\real X) $の有界性を判定しようとする\Cref{prob:1121}にはこのような表現論的背景もある.


\clearpage

\section{設定と$\ha$射影の基本的な性質および\Cref{prob:1121}の観察}
%%% Local Variables:
%%% mode: japanese-latex
%%% TeX-engine: uptex
%%% TeX-master: "okuda-master-thesis"
%%% TeX-PDF-mode: t
%%% TeX-PDF-from-DVI: "Dvipdfmx"
%%% End:

\subsection{記号の設定}
本論文の基本的な設定は次のとおりであり,この他に必要な条件は都度明示することとする.

\begin{nttdef}
  \leavevmode\vspace{-1em}
  \begin{itemize}
  \item $\nat,\real, \cpx,\quat$をそれぞれ0を含む自然数全体,実数全体,複素数全体,四元数全体の集合とする.
  \item $G$を非コンパクト実半単純Lie 群,$H$を$G$の非コンパクトな部分Lie群で,$G$のCartan対合$\Theta$に対して$\Theta H = H$なるものとする.
  \item $\ge \defeq \Lie G,\; \ha \defeq \Lie H$とし,$\ge = \ka\oplus \pe$を $\theta \defeq d\Theta$ によるCartan分解とする.
  \item  $e$を$G$の単位元とし,$o_K \defeq eK\in G/K$とする.
  \item $B({-}, {-}) $を$\ge$のKilling形式とし,$\per{\ha}\defeq \{W\in \ge\mid B(W, \ha) = \{0\} \} $とする.
  \item $X\in \pe$に対し,ベクトル空間としての分解$\pe =( \ha\cap \pe)\oplus(\per{\ha}\cap \pe) $に対応した分解を$X = X_1 + X_2 $,$X_1 \in \ha\cap \pe$,$X_2\in \per{\ha}\cap \pe$とする.
  \end{itemize}  
\end{nttdef}

以下の\Cref{thm:kob89-lem6.1}を用いて,$X\in \pe$に対し,$(Y(X), Z(X))\defeq \inv{\pi}(e^X\cdot o_K)\in (\ha\cap\pe)\oplus (\per{\ha}\cap \pe)$と定義する.
\begin{thm}(\cite[Lemma~6.1]{kob89})\label{thm:kob89-lem6.1}
  $\pi\colon  (\ha\cap\pe)\oplus (\per{\ha}\cap \pe) \ni (Y, Z)\mapsto e^{Y}e^{Z}\cdot o_K \in G/K $は上への微分同相である.
\end{thm}



ここで,$Y(\real X) $の有界性について,次の\Cref{yosou:1121}が小林俊行氏によって立てられた.
\begin{defi}
  $\pe_{H,\bdd}\defeq \{X\in \pe\mid Y(\real X)\text{ が } \ha\cap \pe \text{ の有界集合である.}  \}  $と定める.
\end{defi}

\begin{yosou}(by T.~Kobayashi)\label{yosou:1121}  
  $\pe_{H,\bdd} = \{X\in \pe\mid [X_1, X_2]\neq 0 \text{ あるいは } X_1 = 0\text{ である.} \}$である.
\end{yosou}

\Cref{yosou:1121}についての基本的な事項を挙げる.

\begin{lem}\label{lem:basic-yosou}
  \leavevmode\vspace{-1em}
  \begin{enumerate}
  \item $\pe_{H,\bdd} \subset \{X\in \pe\mid [X_1, X_2]\neq 0 \text{ あるいは } X_1 = 0 \}$である.
  \item $X \in \pe $が$X_1 = 0$を満たすならば$X\in \pe_{H,\bdd} $である.
  \item 1,2より\Cref{yosou:1121}と「$X\in \pe$が$[X_1,X_2]\neq 0$ならば$X\in \pe_{H,\bdd} $である」は同値である.
  \item $G$が実階数1のとき,\Cref{yosou:1121}と「$\pe_{H,\bdd} =  \{0\}\cup \pe\setminus\ha $」は同値である.
  \end{enumerate}
\end{lem}

\begin{npfwn}[\Cref{lem:basic-yosou}]
  \leavevmode\vspace{-1em}
  \begin{enumerate}
  \item 背理法による.$[X_1,X_2 ] = 0$かつ$X_1\neq 0$なる$X  \in \pe $に対しては$[X_1,X_2 ] = 0$より$e^{tX_1}e^{tX_2}\cdot o_K = e^{t(X_1 + X_2)}\cdot o_K = e^{tX}\cdot o_K$である.したがって\Cref{thm:kob89-lem6.1}より$Y(tX) = tX_1 $,$Z(tX) = tX_2 $であることから$Y(\real X) = \real X_1 $となり,$X_1\neq 0$より$Y(\real X)$は有界集合とならない.
  \item $X_1 = 0\iff X\in \per{\ha}\cap\pe $より$Z(tX) = tX $,$Y(tX) = 0 $であることによる.
  \item[4.] 同値な命題である「$X\in \pe$に対し,$[X_1,X_2] = 0 $かつ$X_1 \neq 0$であることと$f X \in \ha\setminus\{0\} $であることは同値である」を示せば良い.$G$の実階数は1で,$H$は非コンパクトかつ$\Theta H = H$であるから,$\ha\subset \pe$であり,$\ha$は$\ge$の極大可換部分空間である.よって$X_1\neq 0$かつ$[X_1,X_2] = 0  \implies X_2 = 0 $であり,$X = X_1 + X_2\in \ha\setminus\{0\} $を得る.
  \end{enumerate}  
\end{npfwn}

$Y(\real X) $の有界性は$\Ad(k) $-不変である.つまり\Cref{lem:1101}が成り立つ.
\begin{lem}\label{lem:1101}
  $k\in K$,$X\in \pe$に対し,$X'\defeq \Ad(k)X $,$\ha'\defeq \Ad(k)\ha $とする.$Y'(X'), Z'(X') $を,微分同相$\pi'\colon (\ha'\cap \pe)\oplus (\per{\ha'}\cap \pe)\ni (Y',Z')\mapsto e^{Y'}e^{Z'}\cdot o_K  $を用いて,$X'\in \pe$に対し,$(Y'(X'), Z'(X')) = \inv{\pi'}(e^{X'}\cdot o_K) $と定めると,$Y(\real X)$が有界であることと$ Y'(\real X') $が有界であることは同値である.
\end{lem}

\begin{npfwn}[\Cref{lem:1101}]
  主張は$(X,\ha) $と$(X',\ha')$に対して対称的であるから,$Y(\real X) $が有界ならば$Y'(\real X') $が有界であることのみを示せば十分である.

  任意に$r\in \real$を取る.$e^{rX'}\cdot o_K = e^{Y'(r X')}e^{Z'(r X')}\cdot o_K  $であり,両辺に左から$\inv{k} $を掛けると,$e^{r X} = e^{\Ad(\inv{k})( Y'(r X'))}e^{\Ad(\inv{k})( Z'(r X'))}\cdot o_K  $を得る.ここで$Y'(rX')\in \ha'\cap \pe $,$Z'(r X')\in \per{\ha'}\cap \pe $であるから$\Ad(\inv{k})(Y'(r X'))\in \ha\cap \pe $,$\Ad(\inv{k})(Z'(r X')) \in \per{\ha}\cap \pe $である.

  \Cref{thm:kob89-lem6.1}により$\pi$は微分同相であるから任意の$r\in \real$に対して$\Ad(\inv{k})(Y'(r X')) = Y(rX)  $である.$Y'(\real X) = \Ad(k)(Y(\real X))  $であり,$\Ad(k) $は有限次元空間の間の線型写像であるから有界性を保つ.

  以上から\Cref{lem:1101}が示された.  
\end{npfwn}


$Z(\real X) $の有界性については次の定理が知られており,有界性の判定はLie環の言葉のみで行える.

\begin{thm}(\cite[Lemmma~5.4]{kob97})\label{thm:kob97}  
  $X\in \pe$に対し,$\norm{Z(X)}\geq \norm{X} \sin\phi(X, \ha\cap\pe)$である.

  ここに$\phi(X,\ha\cap \pe) $は$X$と$\ha\cap \pe$の元がなす角度の最小値$0\leq \phi(X,\ha\cap \pe) \leq \frac{\pi}{2} $であり,
  $X\in \pe\setminus \ha \iff \phi(X,\ha\cap \pe)\neq 0 $である.
\end{thm}

つまり$ X\in \pe\setminus \ha$ならば$\norm{Z(t X)}\to \infty $,$\abs{t}\to \infty $である.


\subsection{\Cref{yosou:1121}の観察: $G = \SU(1,1) $,$H = \SO(1,1) $の場合}

$G = \SU(1,1) $,$H = \SO(1,1) \defeq\lbig\{
\begin{pmatrix}
  \cosh t & \sinh t\\ \sinh t & \cosh t
\end{pmatrix}
\relmiddle| t\in \real \rbig\} $の場合に\Cref{yosou:1121}が正しいことは直接計算により確かめられる.

\begin{prop}\label{prop:yosou-eg}
  $G = \SU(1,1) $,$H = \SO(1,1) $のとき\Cref{yosou:1121}は正しい.
\end{prop}

% \bluetext{$\sulie(1,1) $のKilling形式と$r = \tanh t$の関係を明記せよ.}

\begin{lem}\label{lem:riem-metric-su11}
  $\ge\defeq \sulie(1,1)$のKilling形式から定まる{\Poincare}円板$G/K =\{x+\sqrt{-1}y\mid x^2 + y^2 < 1 \} $の計量は$ \dfrac{8(dx^2 + dy^2)}{(1 - x^2 - y^2)^2} $である.
\end{lem}

\begin{npfwn}[\Cref{lem:riem-metric-su11}]
  
$\ge$の元を$G/K$上の左不変ベクトル場と同一視すると$X' \defeq 
\begin{pmatrix}
  0 & 1 \\ 1 & 0
\end{pmatrix} = \dfrac{\del}{\del x}
$,$Y' \defeq 
\begin{pmatrix}
  0 & \sqrt{-1} \\ -\sqrt{-1} & 0
\end{pmatrix} = \dfrac{\del}{\del y}
$である.$\ge$のKilling形式$B$から定まる$\pe$上のノルム$\norm{-} $に対して$\norm{X'}^2 = \norm{Y'}^2 = 8 $,$B(X', Y' ) = 0$であって,$0\in G/K =\{x+\sqrt{-1}y\mid x^2 + y^2 < 1 \}  $で主張が成り立つ.

したがって$k_{\theta} \defeq \diag(e^{\sqrt{-1}\theta},e^{-\sqrt{-1}\theta}) $,$a_r\defeq
\begin{pmatrix}
  \cosh r & \sinh r \\  \sinh r & \cosh r
\end{pmatrix}
$とすると,
\begin{align*}
  &g(d\tau(k_{\theta/2}a_r)(d\tau(k_{-\theta/2})X'), d\tau(k_{\theta/2}a_r)(d\tau(k_{-\theta/2})X')) \\
  =&\ g (d\tau(k_{\theta/2}a_r)(d\tau(k_{-\theta/2})Y'), d\tau(k_{\theta/2}a_r)(d\tau(k_{-\theta/2})Y')) \\
  =&\ 8, \\
  &g(d\tau(k_{\theta/2}a_r)(d\tau(k_{-\theta/2})X'), d\tau(k_{\theta/2}a_r)(d\tau(k_{-\theta/2})Y'))  = 0
\end{align*}
なるような計量$g $がKilling形式から誘導される計量であるが,それが主張の形であることを示せば良い (これらのベクトルが何を表しているかは\Cref{fig:riem-metric-su11}参照).

\begin{figure}[H]
  \centering
  % \raggedleft
  % \raggedrightp
  \includegraphics[scale=0.08]{../graph/riem-su11.png}
  % \includegraphics[scale=0.3]{../graph/y-and-z.pdf}
  \caption{}
  \label{fig:riem-metric-su11}
\end{figure}


$t = 0$での接ベクトルが$d\tau(k_{\theta/2}a_r)d\tau(k_{-\theta/2})X'$を与える曲線は
\begin{align*}
  \gamma_x(t) \defeq  e^{\sqrt{-1} \theta}\dfrac{\cosh r\cdot e^{-\sqrt{-1}\theta} \tanh t + \sinh r }{\sinh r\cdot e^{-\sqrt{-1}\theta} \tanh t + \cosh r}
\end{align*}
であるから,
\begin{align*}
  \lbig.\dfrac{d}{dt}\rbig|_{t=0}\gamma_x(t) = d\tau(k_{\theta/2}a_r)d\tau(k_{-\theta/2})X' = (1 - \tanh^2 r)\dfrac{\del}{\del x} = (1-x^2-y^2)\dfrac{\del}{\del x}
\end{align*}
である.

同様に$t = 0$での接ベクトルが$d\tau(k_{\theta/2}a_r)d\tau(k_{-\theta/2})Y'$を与える曲線は
\begin{align*}
\gamma_y(t) \defeq  e^{\sqrt{-1} \theta}\dfrac{\cosh r\cdot e^{-\sqrt{-1}\theta}\sqrt{-1} \tanh t + \sinh r }{\sinh r\cdot e^{-\sqrt{-1}\theta}\sqrt{-1} \tanh t + \cosh r}
\end{align*}
であるから,
\begin{align*}
\lbig.\dfrac{d}{dt}\rbig|_{t=0}\gamma_y(t) = d\tau(k_{\theta/2}a_r)d\tau(k_{-\theta/2})Y' = (1 - \tanh^2 r)\dfrac{\del}{\del y} = (1-x^2-y^2)\dfrac{\del}{\del y}
\end{align*}
である.

以上より$g  =  \dfrac{8(dx^2 + dy^2)}{(1 - x^2 - y^2)^2} $が示された.
\end{npfwn}


\begin{npfwn}[\Cref{prop:yosou-eg}]


  $k_{\theta} \defeq \diag(e^{\sqrt{-1}\theta},e^{-\sqrt{-1}\theta}) $,$X_{\theta} \defeq k_{\theta/2}
  \begin{pmatrix}
    0 & 1 \\ 1 & 0
  \end{pmatrix}
  k_{-\theta/2}$とすると,$\pe\setminus\{0\} =  \{tX_{\theta}\mid t\in \real_{>0},\ 0\leq \theta\leq \pi\}$である.この$X_{\theta} $と$t\in \real$に対して$Y(tX_{\theta} ) = s
  \begin{pmatrix}
    0 & 1 \\ 1 & 0
  \end{pmatrix}
  $なる$s\in \real $を求める.


   
  \begin{figure}[H]
    \centering
    % \raggedleft
    % \raggedrightp
    \includegraphics[scale=0.08]{../graph/yosou-eg-1.jpg}
    % \includegraphics[scale=0.3]{../graph/y-and-z.pdf}
    \caption{}
    \label{fig:yosou-eg-1}
  \end{figure}

  右の円の Euclid 距離での半径を$R$とし,$e^{tX_{\theta}}\cdot o_K $から$H\cdot o_K$への垂線の足の$o_K$からの Euclid 距離を$h$とするとき,外側の青色の直角三角形に対して三平方の定理を用いて$(h+R)^2 = R^2 +  1 $より$R = \frac{1-h^2}{2h} , R+h = \frac{1+h^2}{2h}  $を得る.

  さらに下の紫色の三角形に対して余弦定理を用いて$R^2 = (R+h)^2 + r^2 - 2(R+h) \cos\theta  $を得,
  \begin{align}
    {\dfrac{2r\cos\theta}{r^2 + 1} = \dfrac{2h}{h^2 + 1} }\label{eq:1018-main}
  \end{align}
  を得る.

  \bluetext{要確認: ここで\Cref{lem:riem-metric-su11}より$\dfrac{r}{2\sqrt{2}} = \tanh {2\sqrt{2}}t$,$\dfrac{h}{2\sqrt{2}} = \tanh {2\sqrt{2}}s$であり\Cref{eq:1018-main} は$\cos\theta \tanh \dfrac{t}{4\sqrt{2}} = \tanh \dfrac{s}{4\sqrt{2}} $と書き直せる.したがって$X_{\theta}$に対して$Y(\real X) $が有界$\iff \abs{\cos\theta}\neq 1 \iff  X\nin \ha  $である.}
\end{npfwn}

\begin{rem}\label{rem:su11-by-angle}

  \Cref{prop:yosou-eg}は角度を用いた議論によっても示すことができる.具体的には,計算により次の\Cref{lem:0106}が示せる.
  \begin{lem}\label{lem:0106}
    $e^{tY}e^{s Z}\cdot o_K =
    \begin{pmatrix}
      \cosh t & \sinh t\\ \sinh t & \cosh t
    \end{pmatrix}
    \sqrt{-1}\tanh s \in \SU(1,1)/\U(1) $,$t > 0$,$s\in \real$に対し,$o_K $と$e^{tY}e^{sZ}\cdot o_K$を結ぶ測地線が$o_K$と$e^{tY}\cdot o_K $を結ぶ測地線と$o_K$でなす角$\phi_{s,t} $は,$\tan \phi_{s,t} = \dfrac{\tanh 2s}{\sinh 2t} $を満たす.
  \end{lem}
  ただし,$Y \defeq
  \begin{pmatrix}
    0 & 1\\ 1 & 0
  \end{pmatrix}
  $,$Z \defeq \begin{pmatrix}
    0 & \sqrt{-1} \\ -\sqrt{-1} & 0
  \end{pmatrix}$とする.

  \Cref{lem:0106}により\Cref{prop:yosou-eg}は次のように証明できる.
  任意の$s\in \real, 0\neq t\in \real $に対し,
  \begin{align*}
    \lim_{s\to -\infty}\tan \phi_{s,\abs{t}} = \dfrac{-1}{\sinh 2\abs{t}}  \leq \tan \phi_{s,t} \leq  \lim_{s\to \infty}\tan \phi_{s,\abs{t}} = \dfrac{1}{\sinh 2\abs{t}}
  \end{align*}
  であるから,$X\nin \real Y $の元に対して$Y(\real X) $が非有界であるとすると,$\phi(X,\ha) >  \epsilon > 0$なる$\epsilon$に対し,ある$r\in \real $が存在して,$Y(rX) = tY $,$Z(rX) = sZ $に対し$\abs{\tan \phi_{s,t}} < \tan \epsilon $となり,$o_K$と$e^{tY}e^{sZ}\cdot o_K$を結ぶ測地線が$o_K$と$e^{tY}\cdot o_K $を結ぶ測地線と$o_K$でなす角が$\epsilon$未満,つまり$\ha$と$X$の角度未満となって矛盾する ($o_K$と$e^{tX}\cdot o_K (= e^{tY}e^{sZ}\cdot o_K)  $を結ぶ測地線が$o_K$と$e^{tY}\cdot o_K $を結ぶ測地線と$o_K$でなす角は$\phi(X,\ha)$である).

  
\end{rem}

\begin{cor}\label{cor:yosou-eg}
  $G = \SO(1,n) $,$H = \SO(1,k) $,$1\leq k\leq n-1$に対して\Cref{yosou:1121}は正しい.
\end{cor}


\begin{npfwn}[\Cref{cor:yosou-eg}]
  $\SO(1,n)/\SO(n)$の開球としての実現を考える.「$e^X\cdot o_K $と$o_K$を結ぶ直線」と$H\cdot o_K$で張られる超平面で$\SO(1,n)/\SO(n)$を切った際の断面を考える.
  \begin{figure}[H]
    \centering
    % \raggedleft
    % \raggedrightp
    % \includegraphics[scale=0.08]{../graph/fig1.jpg}
    \includegraphics[scale=0.1]{../graph/son1.jpg}
    \caption{}
    \label{fig:son1}
  \end{figure}
  
  この断面に現れるのは\Cref{fig:yosou-eg-1}と同じであるから,同様の計算により\Cref{cor:yosou-eg}を得る.
  
\end{npfwn}


\subsection{\Cref{yosou:1121} の観察: \Cref{yosou:1121}の仮定を外した場合の成り立たない例}

\Cref{yosou:1121}と次の\Cref{yosou:1121-2}は同値である.
\begin{yosou}\label{yosou:1121-2}

  $\pe_{H,\bdd} = \{X\in \pe\mid [X,(\ha\cap\pe)]\neq \{0\} \text{ あるいは } X\perp (\ha\cap\pe)\text{ である.}  \}  $
  
\end{yosou}

ここで似た予想として次の$\ha\cap\pe$を$\ha$に置き換えた予想が立てられる.
\begin{yosou}\label{yosou:1101}
  $\pe_{H,\bdd} = \{X\in \pe\mid  [X,\ha]\neq \{0\} \text{ あるいは } X\perp \ha \text{ である.}\}  $
\end{yosou}

しかし\Cref{yosou:1101}には反例が存在する.
\begin{lem}\label{lem:1118-main}
  $G = \SL(3,\real) $,$Y_1\defeq \diag(1,1,-2)$,$Y_2 \defeq \begin{pmatrix}
    0 & 1 & 0\\
    -1 & 0 & 0 \\
    0 & 0 & 0
  \end{pmatrix}$,\\
  $\ha = \real Y_1 \oplus \real Y_2 $,$X = \diag(1,0,-1) $に対し,$[X,\ha] \neq \{0\} $であるが$Y(\real X) = \real Y_1 $であり,非有界である.
\end{lem}

\begin{ncalcof}[\Cref{lem:1118-main}]

  $\ha$は可換Lie環であり,$\ge = \slie(3,\real) $のCartan対合$\theta W \defeq -\trans{W} $に対し$\ha = \theta \ha$である.

  $[X,\ha]\neq 0 $は,$[X, Y_2] \neq 0$より従う.

  ここで$Z_1\defeq \diag(1,-1,0)\in \per{\ha}\cap \pe $であり,任意の$t\in \real$に対し,$e^{2tX} = e^{tY_1}e^{tZ_1} $であるから,$Y(\real X) = \real Y_1 $となり,\Cref{lem:1118-main} が示された.
\end{ncalcof}

\Cref{lem:1118-main}において$X$と$\ha$は,$[X,\ha] \neq \{0\} $だが$[X,(\ha\cap \pe)] = \{0\}$かつ$X\not\perp (\ha\cap \pe) $となるように取った.  

つまり\Cref{yosou:1101}の右辺を次の\Cref{yosou:1101-2}のように少し弱めても\Cref{lem:1118-main}はその反例になっている.
\begin{yosou}\label{yosou:1101-2}
  $\pe_{H,\bdd} = \{X\in \pe\mid  [X,\ha]\neq \{0\} \text{ あるいは } X\perp (\ha\cap\pe) \text{ である.} \}  $
\end{yosou}



\section{$G$の実階数が1の場合}
\subsection{具体例: 実階数1の古典型単純Lie群の場合}
\begin{prop}\label{prop:classical-rank-one}
  $G = \SO(1,n)$,$ \SU(1, n)$,$\Sp(1,n) $,$H = \SO(1,1) $,$n\geq 2$に対して\Cref{prob:1121} は正しい.
\end{prop}

$G = \Sp(1,2) $,$\ha = \real \begin{pmatrix}
    0 & 1 & 0 \\
    1 & 0 & 0\\
    0 & 0 & 0
  \end{pmatrix}$の場合にのみ示す.その他の場合も全く同様の議論である.
\begin{prop}\label{prop:1127-main}
  $G = \Sp(1,2) $,$H = \SO(1,1)$,$X\in \pe$に対し,$Y(\real X) $が有界であることと$ X\in \{0\}\cup \pe\setminus \ha  $であることは同値である.
\end{prop}

ただし,$H$は$G$の左上に入っている.すなわち,$\ha = \real Y $,$Y\defeq \begin{pmatrix}
  0 & 1 & 0 \\
  1 & 0 & 0\\
  0 & 0 & 0
\end{pmatrix}$とする.

\begin{nttdef}\label{nttdef:1127-main}
  
  $\quat$を四元数体とする.${\Sp(1,2)\defeq \{g\in \GL(3,\quat)\mid \bar{\trans{g}}\; I_{1,2}g = I_{1,2}  \}} $,$I_{1,2} \defeq \diag(-1,1,1) $とし,$\Sp(1,2) $の$\quat^3 $への自然表現を,任意の$\begin{pmatrix}
    x_{11} & x_{12} & x_{13}\\ x_{21} & x_{22} & x_{23}\\ x_{31} & x_{32} & x_{33}
  \end{pmatrix}\in \Sp(1,2)$と任意の$\begin{pmatrix}
    a \\ b \\c
  \end{pmatrix}\in \quat^3 $に対し
  \begin{align*}
    \begin{pmatrix}
      x_{11} & x_{12} & x_{13}\\ x_{21} & x_{22} & x_{23}\\ x_{31} & x_{32} & x_{33}
    \end{pmatrix}\cdot
                                                                              \begin{pmatrix}
                                                                                a \\ b \\c
                                                                              \end{pmatrix}
    =
    \begin{pmatrix}
      x_{11}a + x_{12}b + x_{13}c\\ x_{21}a + x_{22}b + x_{23}c\\ x_{31}a + x_{32}b + x_{33}c\\    
    \end{pmatrix}
  \end{align*}
  により定める.
  
  $\Sp(1,2)/(\Sp(1)\times \Sp(2)) \simeq \{(z_1, z_2)\mid z_1,z_2\in \quat ,\; \abs{z_1}^2 + \abs{z_2}^2   < 1 \} =: \quat\mathbb{H}^2 $である.この$\Sp(1,2)/(\Sp(1)\times \Sp(2)) $と$\quat\mathbb{H}^2$の間の微分同相は$\trans{(1,0,0)} $の$\Sp(1,2)$軌道上の点$
  \begin{pmatrix}
    \zeta_0 \\ \zeta_1 \\ \zeta_2 
  \end{pmatrix}
  \in \quat^3$に対して$\quat\mathbb{H}^2$の点$\begin{pmatrix}
    \zeta_1\inv{\zeta_0} \\ \zeta_2 \inv{\zeta_0}
  \end{pmatrix}$を対応させることで得られる.

  $\trans{(1,0,0)} $の$\Sp(1,2)$軌道上の点$
  \begin{pmatrix}
    \zeta_0 \\ \zeta_1 \\ \zeta_2 
  \end{pmatrix}
  \in \quat^3 $に対応する$\quat\mathbb{H}^2$の点を$\lbig[
  \begin{pmatrix}
    1 \\ \zeta_1\inv{\zeta_0} \\ \zeta_2 \inv{\zeta_0}
  \end{pmatrix}\rbig]
  $と書く.
  
  % により対応させることで得られる.
  % $\trans{(1,0,0)} $の自然表現$\Sp(1,2)\leftaction \quat^3 $による軌道を考え,第2,第3成分に第1成分の逆数を右からかけた空間が$\quat\mathbb{H}^2$と微分同相であるためである.対応すると書く.
\end{nttdef}

行列計算により,次が示される.
\begin{lem}\label{lem:exp-quat}
  
  任意の$ z,w\in \quat$に対し,
  \begin{align*}
    \exp
    \begin{pmatrix}
      0 & z & w  \\
      \bar{z} & 0 & 0\\
      \bar{w} & 0 & 0
    \end{pmatrix}
                    =
                    \begin{pmatrix}
                      \cosh r &  \dfrac{z}{r} \sinh r & \dfrac{w}{r}\sinh r \\
                      \\
                      \dfrac{\bar{z}}{r} \sinh r & \dfrac{\abs{z}^2}{r^2} \cosh r & -1+ \dfrac{\bar{z}w}{r^2}\cosh r \\
                      \\
                      \dfrac{\bar{w}}{r}\sinh r &  -1+ \dfrac{\bar{w}z}{r^2}\cosh r & \dfrac{\abs{w}^2}{r^2} \cosh r
                    \end{pmatrix}
  \end{align*}
  である.ただし$r \defeq \sqrt{\abs{z}^2 + \abs{w}^2 } $とする.
\end{lem}

\begin{npfwn}[\Cref{prop:1127-main}]
  
  % $\Lie H = \ha = \real A $,$A\defeq \begin{pmatrix}
  %   0 & 1 & 0 \\
  %   1 & 0 & 0\\
  %   0 & 0 & 0
  % \end{pmatrix}$とする.
  $X = 0$ならば$ Y(\real X) = \{0\} $である.また$X\in \ha\setminus\{0\} $のときに$Y(\real X) $が非有界であることは明らかであるから,$X\nin \ha $の場合にのみ議論すればよい.したがって$\abs{z_0}^2 +\abs{w_0}^2  = 1 $かつ$ z_0 \neq \pm 1 $を満たす$z_0,w_0\in \quat$により$X =  \begin{pmatrix}
    0 & z_0 & w_0 \\
    \bar{z_0} & 0 & 0 \\
    \bar{w_0} & 0 & 0 
  \end{pmatrix}$と書かれている$X$に対して$Y(\real X) $の有界性を議論して一般性を失わない ($\abs{z_0}^2 +\abs{w_0}^2  = 1 $を仮定したとき$X\nin \ha $であることと$z_0\neq \pm 1$であることは同値である).以下,$X$はこのように書ける元とする.% つまり
  % $X =
  % \begin{pmatrix}
  %   0 & z & w \\
  %   \bar{z} & 0 & 0 \\
  %   \bar{w} & 0 & 0 
  % \end{pmatrix}
  % \in \pe\setminus\ha $,$z,w\in \quat \st \abs{z}^2 +\abs{w}^2  = 1 $を任意に1つ固定して議論して一般性を失わない.このとき,$X\in\pe\setminus \ha $より$\re z \neq \pm 1$であることに注意する ($\re\colon \quat\ni a+bi+cj+dk\mapsto a\in \real$とする).

  $G$の Cartan 対合を$\Theta(g) = \inv{(g^{*})} $ ($g^{*}$は$g$の共役転置) とするとき,$\Theta(e^{Y(tX)}e^{Z(tX)})\cdot o_K = e^{-Y(tX)}e^{-Z(tX)}\cdot o_K = \Theta(e^{X})\cdot o_K = e^{-X}\cdot o_K $より,$Y(\real X) $が非有界であることと$ Y(\real X)\subset \real Y $が上に非有界であることは同値である.

  したがって,$Y (\real X) $が非有界であるとき,必要なら部分列を取り,$Y$の符号を入れ替えて,列$\{t_n \in \real_{\geq 0} \}_{n\in \nat} $で,$t_n\to \infty$かつ$s_n\to \infty$,$ n\to \infty$,ただし$Y(t_n X) = s_n Y$,なるものが存在する.

  任意の$\per{\ha}\cap\pe $の元はある$Z =
  \begin{pmatrix}
    0 & z & w \\
    \bar{z} & 0 & 0 \\
    \bar{w} & 0 & 0 
  \end{pmatrix} \in \per{\ha}\cap\pe $ (ただし$z,w\in \quat$は$ \abs{z}^2 +\abs{w}^2  = 1 $を満たす) と$r\in \real$により$rZ$と表せる.よって$Z(t_nX) = r_nZ_n$,$Z_n\defeq \begin{pmatrix}
    0 & z_n & w_n \\
    \bar{z_n} & 0 & 0 \\
    \bar{w_n} & 0 & 0 
  \end{pmatrix} $ (ただし$z_n,w_n\in \quat $は$ \abs{z_n}^2 +\abs{w_n}^2  = 1 $を満たし,$r_n\in \real$とする) の形で表わせる.$X\nin\ha$であるから\Cref{thm:kob97}より$\abs{r_n}\to \infty $,$n\to \infty$である.$z_n,w_n\in \quat $は$ \abs{z_n}^2 +\abs{w_n}^2  = 1 $を満たすから,$\{t_n\} $の部分列を取るとある$Z_{\infty} $が存在して$Z_{\infty}\defeq \lim_{n\to \infty}Z_n  =
  \begin{pmatrix}
    0 & z_{\infty} & w_{\infty} \\
    \bar{z_{\infty}} & 0 & 0 \\
    \bar{w_{\infty}} & 0 & 0 
  \end{pmatrix}
  \in \per{\ha}\cap\pe $,$\abs{z_{\infty}}^2 +\abs{w_{\infty} }^2  = 1$なるようにできる.$Z\in\pe\setminus \ha $より$\re z_{\infty} \neq \pm 1$であることに注意する ($\re\colon \quat\ni a+bi+cj+dk\mapsto a\in \real$とする).

  \Cref{lem:exp-quat}より, 
  \begin{align*}
    e^{s_n Y}e^{r_n Z_n}\cdot o_K &=
    \begin{pmatrix}
      \cosh s_n & \sinh s_n & 0 \\
      \sinh s_n & \cosh s_n & 0 \\
      0 & 0 & 1 
    \end{pmatrix}
              \lbig[\begin{pmatrix}
                1\\ \pm \bar{z_n} \tanh \abs{r_n}  \\ \pm \bar{w_n} \tanh \abs{r_n}
              \end{pmatrix}\rbig]\\
    &=  \lbig[ \begin{pmatrix}
      \cosh s_n \pm \bar{z_n} \tanh \abs{r_n} \sinh s_n \\ \sinh s_n \pm \bar{z_n} \tanh \abs{r_n} \cosh s_n \\ \pm \bar{w_n} \tanh \abs{r_n}
    \end{pmatrix}\rbig]
  \end{align*}
  である.ただし複号は$r_n$の符号$\pm$と同順である.このとき$\lim_{n\to \infty}\tanh s_n = 1 = \lim_{n\to \infty}\tanh \abs{r_n} $と$\lim_{n\to \infty} \re z_n = \re z_{\infty} \neq \pm 1$に注意すると次を得る.
  \begin{align}
    \lim_{n\to \infty}(\sinh s_n \pm \bar{z_n} \tanh \abs{r_n} \cosh s_n)\inv{(\cosh s_n \pm \bar{z_n} \tanh \abs{r_n} \sinh s_n) } = 1\label{eq:-1}
  \end{align}
  である.\Cref{eq:-1}を得るための具体的な計算は後述する.

  \Cref{eq:-1}より$
  \begin{pmatrix}
    0 \\ 0 
  \end{pmatrix}
  \in \quat\mathbb{H}^2 $から$
  \begin{pmatrix}
    1 \\ 0 
  \end{pmatrix}
  \in \quat\mathbb{H}^2 $へのベクトルと,$
  \begin{pmatrix}
    0 \\ 0 
  \end{pmatrix}
  \in \quat\mathbb{H}^2 $から\\
  $ \begin{pmatrix}
    (\sinh s_n \pm \bar{z_n} \tanh \abs{r_n} \cosh s_n)\inv{(\cosh s_n \pm \bar{z_n} \tanh \abs{r_n} \sinh s_n) } \\  \ast 
  \end{pmatrix}\in \quat\mathbb{H}^2 $へのベクトルがなすEuclid内積の値を $I_n$とすると,$\lim_{n\to \infty}I_n = 1 $である.

  $
  \begin{pmatrix}
    0 \\ 0 
  \end{pmatrix}
  \in \quat\mathbb{H}^2 $から$
  \begin{pmatrix}
    1 \\ 0
  \end{pmatrix}
  \in \quat\mathbb{H}^2 $へのベクトルと,$
  \begin{pmatrix}
    0 \\ 0 
  \end{pmatrix}
  \in \quat\mathbb{H}^2 $から$e^{t_nX}\cdot o_K \in \quat \mathbb{H}^2 $へのベクトルがなすEuclid内積の値$J_n$を計算する.$X =   \begin{pmatrix}
    0 & z_0 & w_0 \\
    \bar{z_0} & 0 & 0 \\
    \bar{w_0} & 0 & 0 
  \end{pmatrix}$ (ただし$\abs{z_0}^2 + \abs{w_0}^2 = 1 $かつ$z_0\neq \pm 1 $) と設定したことを思い出すと,$J_n = \bar{z_0}\tanh t_n $である.$X\nin \ha $であることと$z_0\neq \pm 1$であることは同値であることと$t_n\to \infty$,$n\to \infty$より$\lim_{n\to \infty}J_n = \bar{z_0}\neq 1 $である.% ,$r_0\defeq \sqrt{\abs{z_0}^2 + \abs{w_0}^2 } $

  $e^{s_n Y}e^{r_n Z_n}\cdot o_K = e^{t_n X}\cdot o_K$より$ \lim_{n\to \infty} I_n = \lim_{n\to \infty}J_n  $であるが,以上2つの議論を合わせると$\lim_{n\to \infty} I_n= 1$,$\lim_{n\to \infty}J_n \neq 1 $であるから矛盾する.
  

  以上より$X\in \pe\setminus\ha$ならば$Y(\real X) $有界であり,\Cref{prop:1127-main} を得る.  
\end{npfwn}


\begin{ncalcof}[\Cref{prop:1127-main},\Cref{eq:-1}]
    
  \begin{align*}
    \lim_{n\to \infty}\abs{(\sinh s_n \pm \bar{z_n} \tanh \abs{r_n} \cosh s_n)\inv{(\cosh s_n \pm \bar{z_n} \tanh \abs{r_n} \sinh s_n) } - 1} = 0
  \end{align*}
  を示せば主張が得られる.具体的に計算すると,
  \begin{align}
    &\lim_{n\to \infty}\abs{(\sinh s_n \pm \bar{z_n} \tanh \abs{r_n} \cosh s_n)\inv{(\cosh s_n \pm \bar{z_n} \tanh \abs{r_n} \sinh s_n) } - 1}\notag \\
    % &\qquad\qquad\qquad \text{($\text{(この極限)} = 0$を示せば良い)} \\
    = & \lim_{n\to \infty}\lbig|\frac{(\tanh s_n\pm \bar{z_n}\tanh \abs{r_n})(1 \pm z_n \tanh\abs{r_n}\tanh s_n )}{\abs{(1 \pm \bar{z_n} \tanh\abs{r_n}\tanh s_n )}^2} - 1 \rbig|\label{eq:ast} \tag{$\ast$}  
  \end{align}
  である.ここで$ z_n'\defeq 1 \pm z_n\tanh\abs{r_n}\tanh s_n$とおくと,
  \begin{align*}
    (\ast)= & \lim_{n\to \infty}\frac{\abs{(\tanh s_n\pm \bar{z_n}\tanh \abs{r_n})z_n' - (1 \pm \bar{z_n} \tanh\abs{r_n}\tanh s_n )z_n' }}{\abs{(1 \pm \bar{z_n} \tanh\abs{r_n}\tanh s_n )}^2}\\
            &= \lim_{n\to \infty}\frac{\abs{(\tanh s_n\pm \bar{z_n}\tanh \abs{r_n})z_n' - (1 \pm \bar{z_n} \tanh\abs{r_n}\tanh s_n )z_n' }}{\abs{(1 \pm \bar{z_n} \tanh\abs{r_n}\tanh s_n )}^2}\\
    = & \lim_{n\to \infty}\frac{\abs{(1-\tanh s_n)(-1\pm \bar{z_n}\tanh \abs{r_n} )z_n'}}{\abs{(1 \pm \bar{z_n} \tanh\abs{r_n}\tanh s_n )}^2} \\
    = & \lim_{n\to \infty}\frac{\abs{(1-\tanh s_n)(-1\pm \bar{z_n}\tanh \abs{r_n} )}}{\abs{(1 \pm \bar{z_n} \tanh\abs{r_n}\tanh s_n )}} 
  \end{align*}
  であり,
  $0 < \min{\abs{1\pm \re z_n}}\leq  \abs{(1 \pm \bar{z_n} \tanh\abs{r_n}\tanh s_n )}\leq \sqrt{2^2 + 1^2} = \sqrt{5} $と$\min\{\abs{-1\pm \re z_n } \}  \leq \abs{-1\pm\bar{z_n}\tanh \abs{r_n} } \leq \sqrt{5} $であることと$\lim_{n\to \infty} \re z_n = \re z_{\infty} \neq \pm 1$より,
    \begin{align*}
      0 &= \lim_{n\to \infty}(1-\tanh s_n)\frac{\min\{\abs{-1\pm \re z_n} \}}{\sqrt{5}} \\
        &\leq \lim_{n\to \infty}\frac{\abs{(1-\tanh s_n)(-1\pm \bar{z_n}\tanh \abs{r_n} )}}{\abs{(1 \pm \bar{z_n} \tanh\abs{r_n}\tanh s_n )}}\\
      &\leq \lim_{n\to \infty}(1-\tanh s_n)\frac{\sqrt{5}}{\min\{\abs{1\pm \re z_n} \}} = 0
    \end{align*}
    を得,\Cref{eq:-1} が成り立つ.
\end{ncalcof}


%%% Local Variables:
%%% mode: latex
%%% TeX-master: "okuda-master-thesis"
%%% End:

\subsection{$ G$の実階数が1の場合}

\begin{thm}\label{thm:1216-main}
  $G$を実階数1の実線型半単純Lie群,$H$を$G$の非コンパクトかつ連結成分有限個の閉部分群で,$G$のCartan対合$\Theta$に対して$\Theta H = H$なるものとする.さらに$\dim \ha\cap\pe = 1$なるとき,$\pe_{H,\bdd} = \{0\}\cup\pe\setminus\ha $である.
\end{thm}

\begin{thm}\label{thm:0810}(\cite[p.~409, Theorem~3.1]{hel01},$\SU(2,1) $-reduction)
  $\ge = \ka \oplus \pe$を実半単純Lie環$\ge$のCartan対合$\theta$に対する Cartan分解とし,ある$\alpha\in \ah^{*}\setminus\{0\} $に対して$\alpha,2\alpha\in \Sigma(\ge,\ah) $と仮定する.$0\neq X_{\alpha}\in \ge_{\alpha} $,$0\neq X_{2\alpha}\in \ge_{2\alpha} $を任意に固定したとき,$X_{\alpha},X_{2\alpha}, \theta X_{\alpha}, \theta X_{2\alpha} $から生成されるLie環$\ge^{*} $は$\sulie(2,1)$と同型である.
  
\end{thm}
以下で\Cref{thm:0810}を示すための補題や記号を設定し,\Cref{thm:0810}を示す.
\begin{nttdef}\label{nttdef:su21-red}
  \leavevmode
  \vspace{-1em}
  \begin{itemize}
  \item $\ah \subset \ge$を極大分裂可換部分代数,$\emm \defeq \zet_{\ka}(\ah) \defeq \{W\in \ka\mid [W,\ah] = \{0\} \} $とする.$B$を$\ge$のKilling 形式とする.
  \item $\Sigma(\ge,\ah) $を$\ah$に関する制限ルート系とする.$\ge_{\lambda} $を$\lambda\in \ah^{*}$に対応するルート空間とする.
  \item $Y_{\alpha}\defeq [\theta X_{\alpha}, X_{2\alpha}] $とする.
  \item $A_{\alpha}\in \ah $を,任意の$H\in \ah$に対して$B(H,A_{\alpha}) = \alpha(H) $を満たす元とする.
    
    このとき,任意の$H\in \ah$に対して$B(H, [X_{\alpha}, \theta X_{\alpha}]) = \alpha(H) B(X_{\alpha}, \theta X_{\alpha}) $である.したがって
      \begin{align*}
        [X_{\alpha}, \theta X_{\alpha}] &= B(X_{\alpha}, \theta X_{\alpha})A_{\alpha} ,\\
        [Y_{\alpha}, \theta Y_{\alpha}] &= B(Y_{\alpha}, \theta Y_{\alpha})A_{\alpha}, \\
        [X_{2\alpha}, \theta X_{2\alpha}] &= 2B(X_{2\alpha}, \theta X_{2\alpha})A_{\alpha} 
      \end{align*}
    である.
    \item $c_{\alpha}\defeq \sqrt{\dfrac{-2}{\alpha(A_{\alpha})B(X_{\alpha}, \theta X_{\alpha})}} $,$ c_{2\alpha}  \defeq \sqrt{\dfrac{-2}{\alpha(A_{\alpha})B(X_{2\alpha}, \theta X_{2\alpha})}} $とし,
      \begin{align*}
        X_{\alpha}^{*} &\defeq c_{\alpha}X_{\alpha}, \\
        X_{2\alpha}^{*} &\defeq c_{2\alpha}X_{2\alpha},\\ 
        Y_{\alpha}^{*} &\defeq [\theta X_{\alpha}^{*}, X_{2\alpha}^{*} ] = c_{\alpha}c_{2\alpha}Y_{\alpha} ,\\
        A_{\alpha}^{*} &\defeq \dfrac{1}{12\alpha(A_{\alpha}) % \lyama \alpha, \alpha\ryama
      }A_{\alpha} 
      \end{align*}
    とする.
  \end{itemize}
\end{nttdef}

\begin{lem}\label{lem:3.2}
  $c \defeq 2\alpha(A_{\alpha})B(X_{\alpha},\theta X_{\alpha}) $とすると,$[X_{\alpha}, Y_{\alpha}] = cX_{2\alpha} $である.特に$0\neq Y_{\alpha}$,$Y_{\alpha} \neq X_{\alpha} $である.
\end{lem}
\begin{npfwn}[\Cref{lem:3.2}]
  Jacobi恒等式と$Y_{\alpha}$の定義より
  \begin{align*}
    0 &= [X_{\alpha},[\theta X_{\alpha}, X_{2\alpha}]] + [\theta X_{\alpha},[X_{2\alpha}, X_{\alpha}]] + [X_{2\alpha},[X_{\alpha}, \theta X_{\alpha}]] \\
      &= [X_{\alpha}, Y_{\alpha}] + [\theta X_{\alpha},[X_{2\alpha}, X_{\alpha}]]  + [X_{2\alpha},B(X_{\alpha},\theta X_{\alpha})A_{\alpha}]
  \end{align*}
  であり,$3\alpha\nin \Sigma(\ge,\ah) $より第二項が0となることから\Cref{lem:3.2}が従う.
\end{npfwn}

\begin{lem}\label{lem:3.3}
  $[X_{\alpha},\theta Y_{\alpha}]\in \emm\setminus\{0\} $である.また$[[X_{\alpha}, \theta Y_{\alpha}], X_{\alpha}] = -3\alpha(A_{\alpha})B(X_{\alpha}, \theta X_{\alpha})Y_{\alpha} $である.
\end{lem}

\begin{npfwn}[\Cref{lem:3.3}]
  $Y_{\alpha} \in \ge_{\alpha} $より$[X_{\alpha},\theta Y_{\alpha}]\in \emm+ \ah$であり,任意の$H\in \ah$に対して
  \begin{align*}
    B(H, [X_{\alpha},\theta Y_{\alpha}]) &= B([H, X_{\alpha}], Y_{\alpha}) = \alpha(H) B(X_{\alpha}, [X_{\alpha}, \theta X_{2\alpha}]) \\
                                         &= \alpha(H)B([X_{\alpha}, X_{\alpha}], X_{2\alpha}) \\
                                         &= 0
  \end{align*}
  であることより$[X_{\alpha},\theta Y_{\alpha}]\in \emm$である.

  さらに,
  \begin{align*}
    [[\theta X_{\alpha},Y_{\alpha}], X_{\alpha}] &= -[[Y_{\alpha}, X_{\alpha}], \theta X_{\alpha}] - [[X_{\alpha}, \theta X_{\alpha}], Y_{\alpha}]\\
                                                 &= c[X_{2\alpha}, \theta X_{\alpha}] -  B(X_{\alpha},\theta X_{\alpha})\alpha(A_{\alpha})Y_{\alpha}\\
                                                 &= -cY_{\alpha}-  B(X_{\alpha},\theta X_{\alpha})\alpha(A_{\alpha})Y_{\alpha} \\
                                                 &= -3\alpha(A_{\alpha})B(X_{\alpha}, \theta X_{\alpha})Y_{\alpha} \neq 0
  \end{align*}
  より,$\theta [\theta X_{\alpha},Y_{\alpha}] = [X_{\alpha},\theta Y_{\alpha}]\in \emm\setminus\{0\} $である.
\end{npfwn}

\vspace{-1em}
\begin{lem}\label{lem:3.4}  
  $\real X_{\alpha} + \real Y_{\alpha} $は$\ad_{\ge}([X_{\alpha}, \theta Y_{\alpha}]) $で不変である.  
  さらに
  \begin{align*}
    [[X_{\alpha}, \theta Y_{\alpha}], Y_{\alpha}] &= -6\alpha(A_{\alpha})^2B(X_{\alpha}, \theta X_{\alpha})B(X_{2\alpha}, \theta X_{2\alpha})X_{\alpha},\\
  [Y_{\alpha}, \theta Y_{\alpha}] &= -2\alpha(A_{\alpha})B(X_{\alpha}, \theta X_{\alpha})B(X_{2\alpha}, \theta X_{2\alpha})A_{\alpha}
  \end{align*}
  である.

\end{lem}

\begin{npfwn}[\Cref{lem:3.4}]
  $[[X_{\alpha}, \theta Y_{\alpha}], Y_{\alpha}]  \in \real X_{\alpha} $を示せば,\Cref{lem:3.3}と併せて\Cref{lem:3.4}が従う.

  
  \begin{align*}
    [[X_{\alpha}, \theta Y_{\alpha}], Y_{\alpha}] &= -[[\theta Y_{\alpha}, Y_{\alpha}], X_{\alpha}] - [[Y_{\alpha}, X_{\alpha}], \theta Y_{\alpha}] \\
                                                  &= B(Y_{\alpha}, \theta Y_{\alpha})\alpha(A_{\alpha})X_{\alpha} +c[X_{2\alpha},[X_{\alpha}, \theta X_{2\alpha}]] \\
                                                  &= B(Y_{\alpha}, \theta Y_{\alpha})\alpha(A_{\alpha})X_{\alpha} - c[X_{\alpha}, [\theta X_{2\alpha}, X_{2\alpha}]] - c[\theta X_{2\alpha},[X_{2\alpha},X_{\alpha}]] \\
                                                  &= B(Y_{\alpha}, \theta Y_{\alpha})\alpha(A_{\alpha})X_{\alpha} - cB(X_{2\alpha},\theta X_{2\alpha})\alpha(A_{2\alpha})X_{\alpha}
  \end{align*}
  であり ($ \ge_{3\alpha} = \{0\} $による),$A_{2\alpha} = 2A_{\alpha} $であるから,
  \begin{align*}
[[X_{\alpha}, \theta Y_{\alpha}], Y_{\alpha}] =  B(Y_{\alpha}, \theta Y_{\alpha})\alpha(A_{\alpha})X_{\alpha} - 4\alpha(A_{\alpha})^2B(X_{\alpha}, \theta X_{\alpha})B(X_{2\alpha}, \theta X_{2\alpha})X_{\alpha}
  \end{align*}
  を得る.

  さらに,
  \begin{align*}
    B(Y_{\alpha}, \theta Y_{\alpha}) &= B(Y_{\alpha},[X_{\alpha}, \theta X_{2\alpha}]) = -B([X_{\alpha}, Y_{\alpha}], \theta X_{2\alpha}) \\
                                     &= -2\alpha(A_{\alpha})B(X_{\alpha},\theta X_{\alpha})B(X_{2\alpha}, \theta X_{2\alpha})
  \end{align*}
  であるから,最終的に
  \begin{align*}
    [[X_{\alpha}, \theta Y_{\alpha}], Y_{\alpha}] &=  B(Y_{\alpha}, \theta Y_{\alpha})\alpha(A_{\alpha})X_{\alpha} - 4\alpha(A_{\alpha})^2B(X_{\alpha}, \theta X_{\alpha})B(X_{2\alpha}, \theta X_{2\alpha})X_{\alpha} \\
                                                  &= -6\alpha(A_{\alpha})^2B(X_{\alpha}, \theta X_{\alpha})B(X_{2\alpha}, \theta X_{2\alpha})X_{\alpha}
  \end{align*}
を得る.  
\end{npfwn}

\begin{lem}\label{lem:3.5}
  $[[X_{\alpha}, \theta Y_{\alpha}], X_{2\alpha}] = 0$である.
\end{lem}
\begin{npfwn}[\Cref{lem:3.5}]
  \Cref{lem:3.2}--\ref{lem:3.4}とJacobi恒等式による.具体的には,$T\defeq [X_{\alpha},\theta Y_{\alpha}] $とすると,\Cref{lem:3.3}と\Cref{lem:3.4}よりそれぞれ
  \begin{align}
    [T, X_{\alpha}] \in \real Y_{\alpha} ,\ [T, Y_{\alpha}] \in \real X_{\alpha}\label{eq:lem3.5}
  \end{align}
  である.Jacobi恒等式より
  \begin{align}
    [T, [X_{\alpha},Y_{\alpha}]] = -[X_{\alpha},[Y_{\alpha},T]] - [Y_{\alpha},[X_{\alpha},T]]\label{eq:lem3.5-2}
  \end{align}
  であり,\Cref{eq:lem3.5}より\Cref{eq:lem3.5-2}の右辺は0となる.\Cref{lem:3.2}より$[X_{\alpha}, Y_{\alpha}] = cX_{2\alpha} $であり,これを用いることで\Cref{lem:3.5}の主張を得る.
\end{npfwn}

\begin{lem}\label{lem:3.6}
  $[Y_{\alpha},\theta X_{2\alpha}] = 2\alpha(A_{\alpha})B(X_{2\alpha}, \theta X_{2\alpha})\theta X_{\alpha} $である.  
\end{lem}
\begin{npfwn}[\Cref{lem:3.6}]
  Jacobi恒等式を用いて与式を変形し計算することにより主張が示せる.
\end{npfwn}



\begin{npfwn}[\Cref{thm:0810}]
  \begin{align*}
    \ge^{*}_0 &\defeq \real A_{\alpha}\oplus \real[X_{\alpha}, \theta Y_{\alpha}], \\
    \ge^{*}_{\alpha} &\defeq \real X_{\alpha}\oplus \real Y_{\alpha},  \\
    \ge^{*}_{-\alpha} &\defeq \real \theta X_{\alpha}\oplus \real \theta Y_{\alpha}, \\
    \ge^{*}_{2\alpha} &\defeq \real X_{2\alpha}, \\
    \ge^{*}_{-2\alpha} &\defeq \real \theta X_{2\alpha}
  \end{align*}
  とすると,\Cref{lem:3.2}--\ref{lem:3.6}より,$\ge^{*} = \ge^{*}_{0} \oplus \ge^{*}_{\alpha}\oplus \ge^{*}_{-\alpha} \oplus \ge^{*}_{2\alpha} \oplus \ge^{*}_{-2\alpha} $が得られる.

  
  非自明な$\ge^{*} $のLie括弧の関係は以下の通りである (残りの関係式はこの両辺に$\theta$をつけることで得られる).

  
  \begin{align*}
    [X_{\alpha}^{*}, Y_{\alpha}^{*}] &= -4X_{2\alpha}^{*}, &\text{(\Cref{lem:3.2}による)},\\
    [X_{\alpha}^{*},[X_{\alpha}^{*}, \theta Y_{\alpha}^{*}]]  &= -6Y_{\alpha}^{*}, &\text{(\Cref{lem:3.3}による)},\\
    [X_{\alpha}^{*}, \theta X_{\alpha}^{*}] &= -24A_{\alpha}^{*}, & (定義による),\\
    [X_{\alpha}^{*},X_{2\alpha}^{*}] &= 0, &\text{($\ge_{3\alpha} = 0$による)},\\
    [X_{\alpha}^{*}, \theta X_{2\alpha}^{*}] &= \theta Y_{\alpha}^{*} ,& (定義による)\\
    [Y_{\alpha}^{*},X_{2\alpha}^{*}] &= 0, &\text{(\Cref{lem:3.5}による)},\\
    [Y_{\alpha}^{*},\theta X_{2\alpha}^{*}] &= -4\theta X_{\alpha}^{*} , &\text{(\Cref{lem:3.6}による)},\\
    [Y_{\alpha}^{*}, \theta Y_{\alpha}^{*}] &= -96A_{\alpha}^{*} , &\text{(\Cref{lem:3.4}による)},\\
    [Y_{\alpha}^{*}, [X_{\alpha}^{*}, \theta Y_{\alpha}^{*}]] &= 24X_{\alpha}^{*},  &\text{(\Cref{lem:3.4}による)},\\
    [[X_{\alpha}^{*}, \theta Y_{\alpha}], X_{2\alpha}^{*}] &= [[X_{\alpha}^{*}, \theta Y_{\alpha}], \theta X_{2\alpha}^{*}] = 0, &\text{(\Cref{lem:3.6}による)},\\
    [X_{2\alpha}^{*} ,\theta X_{2\alpha}^{*} ] &= -48A_{\alpha}^{*}, & (定義による)
  \end{align*}

  これらを踏まえて$\ge^{*} $と$\sulie(2,1) $の対応を,
  \begin{align*}
    X_{\alpha}^{*} & \leftrightarrow
                     \begin{pmatrix}
                       0 & 1 & 0\\ -1 & 0 & 1\\ 0 & 1 & 0
                     \end{pmatrix},&   X_{2\alpha}^{*}  \leftrightarrow
                     \begin{pmatrix}
                       \img & 0 & -\img \\ 0 & 0 & 0\\ \img & 0 & -\img
                     \end{pmatrix},\\
    \theta X_{\alpha}^{*} & \leftrightarrow
                     \begin{pmatrix}
                       0 & 1 & 0\\ -1 & 0 & -1\\ 0 & -1 & 0
                     \end{pmatrix},&   \theta X_{2\alpha}^{*}  \leftrightarrow
                     \begin{pmatrix}
                       \img & 0 & \img \\ 0 & 0 & 0\\ -\img & 0 & -\img
                     \end{pmatrix},\\
    Y_{\alpha}^{*} & \leftrightarrow
                     -2 \begin{pmatrix}
                       0 & \img & 0\\ \img & 0 & -\img \\ 0 & \img & 0
                     \end{pmatrix},&   \theta Y_{\alpha}^{*}  \leftrightarrow
                     \begin{pmatrix}
                       0 & \img  & 0 \\ \img & 0 & \img \\ 0 & -\img & -0
                     \end{pmatrix},\\
    A_{\alpha}^{*} & \leftrightarrow
                    \frac{1}{12} \begin{pmatrix}
                       0 & 0 & 1\\ 0 & 0 & 0\\ 1 & 0 & 0
                     \end{pmatrix},&   [X_{\alpha},\theta Y_{\alpha}^{*}] \leftrightarrow
                    -4 \begin{pmatrix}
                       \img & 0 & 0 \\ 0 & -2\img & 0\\  & 0 & \img
                     \end{pmatrix}
  \end{align*}
  でつける.この対応がLie環としての同型であること (上の関係式が満たされること) は計算することにより従う.

  以上より\Cref{thm:0810}が示された.  
\end{npfwn}

\begin{lem}\label{lem:su11}
  
  ある$\alpha\in \ah^{*}\setminus\{0\} $が存在して$\Sigma(\ge,\ah) = \{\pm\alpha\}$なる場合,任意に固定した$0\neq X_{\alpha}\in \ge_{\alpha}$と$\theta X_{\alpha}$により生成される部分Lie環$\ge'$は$\sulie(1,1)$と同型である.
\end{lem}

\begin{npfwn}[\Cref{lem:su11}]
  $\ge_{2\alpha} = \ge_{-2\alpha} = \{0\}  $より,$[X_{\alpha}, X_{\alpha}] = [X_{-\alpha}, X_{-\alpha}] = 0 $である. $A_{\alpha}\in \ah $を任意の$H\in \ah$に対して$B(H,A_{\alpha}) = \alpha(H) $を満たす元とする.任意の$H\in \ah$に対して$B(H, [X_{\alpha}, \theta X_{\alpha}]) = \alpha(H) B(X_{\alpha}, \theta X_{\alpha}) $である.任意の$0\neq W\in \ge$に対し$-B(W,\theta W) > 0 $より$[X_{\alpha}, \theta X_{\alpha}] = B(X_{\alpha}, \theta X_{\alpha})A_{\alpha}\neq 0 $である.

  以上より$X_{\alpha} $と$\theta X_{\alpha} $により生成される$\ge$の部分Lie環$\ge'$は$\ge' = \real A_{\alpha}\oplus \real X_{\alpha} \oplus \real X_{-\alpha}  $である.

  % $\tr(\ad_{\ge'}A_{\alpha} \ad_{\ge'}A_{\alpha} ) = 2\alpha(A_{\alpha})^2\neq 0 $より,
  $c_{\alpha}\defeq \dfrac{2}{\alpha(A_{\alpha})} $を用いて$\ge'$と$\sulie(1,1)$の対応を
  \begin{align*}
    % \dfrac{1}{\alpha(A_{\alpha})}
    A_{\alpha} &\leftrightarrow
    \begin{pmatrix}
      0 & 1 \\ 1 & 0
    \end{pmatrix},&  c_{\alpha} X_{\alpha} \leftrightarrow
      \begin{pmatrix}
        \sqrt{-1} & -\sqrt{-1} \\ \sqrt{-1} & -\sqrt{-1}
      \end{pmatrix},\\
    c_{\alpha}X_{-\alpha} &\leftrightarrow
      \begin{pmatrix}
        \sqrt{-1} & \sqrt{-1} \\ -\sqrt{-1} & -\sqrt{-1}
      \end{pmatrix} 
  \end{align*}
  により与えると,これは$\ge'$と$\sulie(1,1)$の間の同型になっている.
\end{npfwn}

\begin{cor}\label{cor:sub-lie-alg}
  $G$を実階数1の実半単純Lie群とする.任意の$0\neq Y\in \pe\cap\ha$と任意の$X\in \pe\setminus \ha$を固定したとき,$X,Y$を含む部分Lie環$\ge_0\subset \ge$で,$\ge_0\simeq \sulie(1,1) $か$\ge_0\simeq \sulie(2,1)$なるものが存在する.
\end{cor}

\begin{npfwn}[\Cref{cor:sub-lie-alg}]
  $G$は実階数1より,極大分裂可換部分代数$\ah\defeq \real Y\subset \ge$に対しある$\alpha\in \ah^{*}\setminus\{0\} $が存在して$\Sigma(\ge,\ah) = \{\pm\alpha\} $あるいは$\Sigma(\ge,\ah)  = \{\pm\alpha, \pm 2\alpha \} $であり,それぞれ\Cref{thm:0810}と\Cref{lem:su11}より\Cref{cor:sub-lie-alg}の主張が従う.以下で$\Sigma(\ge,\ah)$の形で場合分けしてこの議論を確認する.
  \begin{case}
    \textbf{$\Sigma(\ge,\ah) = \{\pm\alpha\} $の場合}

    $\ge = \ge_{0} \oplus \ge_{\alpha}\oplus \ge_{-\alpha} $,$\ge_0 \defeq \ze_{\ge}(\ah) $より$X\in  \pe\setminus \ah$をこの分解に対応して$X = X_0 + X_{\alpha} + X_{-\alpha} $と分解すると,$X\in \pe\setminus \ah$より$ X_{-\alpha} = -\theta X_{\alpha}\neq 0 $である.\Cref{lem:su11}の証明より$Y\in \real [X_{\alpha}, \theta X_{\alpha}] \neq \{0\}$であるからこの$X_{\alpha}\neq 0 $に\Cref{lem:su11}を適用することにより$\ge_0\simeq \sulie(1,1)$で$X,Y\in \ge_0 $なるものが存在する.
    
  \end{case}
  
  \begin{case}
    \textbf{$\Sigma(\ge,\ah) = \{\pm\alpha, \pm 2\alpha\} $の場合}

    $\ge = \ge_{0} \oplus \ge_{\alpha}\oplus \ge_{-\alpha} \oplus \ge_{2\alpha}\oplus \ge_{-2\alpha}  $,$\ge_0 \defeq \ze_{\ge}(\ah) $より$X\in \pe\setminus \ah$をこの分解に対応して$X = X_0 + X_{\alpha} + X_{-\alpha} + X_{2\alpha} + X_{-2\alpha} $と書くと,$X\in \pe$より$ X_{-\alpha} = - \theta X_{\alpha} $,$ X_{-2\alpha} = - \theta X_{2\alpha} $である.

    ここで$X\nin \ah$より,
    \begin{enumerate}
    \item $X_{\alpha}\neq 0 $かつ$X_{2\alpha}\neq 0 $
    \item $X_{\alpha}\neq 0 $かつ$X_{2\alpha} =  0 $
    \item $X_{\alpha} =  0 $かつ$X_{2\alpha} \neq  0 $
    \end{enumerate}
    のいずれかである.
    \begin{enumerate}
    \item[1] の場合はこの$X_{\alpha}, X_{2\alpha} $と$Y$に,
    \item[2] の場合はこの$X_{\alpha}$と,適当な$0\neq X_{2\alpha}'\in \ge_{2\alpha} $と$Y$に,
    \item[3] の場合はこの$X_{2\alpha}$と,適当な$0\neq X_{\alpha}'\in \ge_{\alpha} $と$Y$に,
    \end{enumerate}
    \Cref{thm:0810}を適用することにより$\ge_0\simeq \sulie(2,1)$で$X,Y\in \ge_0 $なるものが存在する.    
  \end{case}   
\end{npfwn}

\Cref{cor:sub-lie-alg}で定めた$\ge_0$とその$G$における解析部分群$G_0$に関係して次の3つが成り立つ.
\begin{lem}(\cite[p.~409, Lemma~2.2]{hel01})
  $\ge$のCartan対合$\theta$に対して$\ge_0 = \theta\ge_0$であり,$\ge_0 $への$\theta$の制限は$\ge_0$のCartan分解を与える.
\end{lem}

\begin{thm}\label{thm:yos38}(\cite[p.~82]{yos38})
  $G_L$を線型Lie群,$\ha_{L} \subset \ge_{L}\defeq \Lie G_{L} $は実半単純な部分Lie環とする.このとき$\ha_{L} $の$G_{L} $における解析的部分群は閉部分群である.
\end{thm}
\begin{cor}
  \Cref{cor:sub-lie-alg}の$\ge_0$の$G$における解析的部分群を$G_0$とする.$G_0 $は$G$の閉部分群である.
\end{cor}


\begin{lem}(\cite[p.~409, Lemma~2.3]{hel01}) \Cref{cor:sub-lie-alg}の$\ge_0$の$G$における解析的部分群を$G_0$とする.$G = KAN$を$G$の岩澤分解,$G_0 = K_0A_0N_0$を$G$の岩澤分解とするとき,
  \begin{align*}
    K_0\defeq G_0\cap K,\ A_0 \defeq G_0\cap A,\ N_0\defeq G_0\cap N, 
  \end{align*}
  であり,$G_0/K_0 \simeq G_0/K$は$G/K$の全測地的な部分Riemann多様体である.
  
\end{lem}

以上のことを用いて,$G$が実階数1の実線型半単純Lie群,$\dim \ha\cap\pe = 1 $の場合を$G= \SU(1,2) $かつ$H = \SO(1,1) $ないし$G = \SU(1,1) $かつ$H = \SO(1,1) $に帰着させることにより\Cref{thm:1216-main}を示す.

\begin{npfwn}[\Cref{thm:1216-main}]% \bluetext{おそらく$\dim H = 1$でない場合も背理法で示せる気がする.}
  $\ge$の極大分裂可換部分代数$\ha\cap\pe$の定める制限ルート系を$\Sigma(\ge,\ha\cap\pe) $とし,$\Sigma(\ge,\ha\cap\pe) $の形によって2通りに場合分けして証明する.
  
  \textbf{ある$\alpha\in (\ha\cap\pe)^{*}\setminus\{0\} $が存在して$\Sigma(\ge,\ha\cap\pe) = \{\pm\alpha \}$なるとき}
  
  \Cref{cor:sub-lie-alg}により$X$と$\ha\cap\pe $を含む部分Lie環$\ge' \subset \ge $で$\sulie(1,1) $に同型なものが存在する.$\ge' $に対応する$G$の解析的部分群を$G'$とし,その岩澤分解を$G' = K'A'N' $とする.このとき$e^{Z(t X)}\cdot o_K = e^{-Y(tX)}e^{tX}\cdot o_K\in G'/K $であるから$Z(tX)\in \ge'\cap \per{\ha}\cap \pe\subset \ge'\cap\pe $であり,$Y(\real X) $の有界性の議論は全測地的的な部分Riemann多様体$G'/K'\subset G/K$に対して行えば良いことがわかる.したがって\Cref{prop:prob-eg}により$X\in\pe_{H,\bdd}$であることと$ X \in\{0\}\cup \pe\setminus\ha $が同値であることが言え,\Cref{thm:1216-main}が示された.
  
  % 任意の固定した$\tau \in \real$に対し$\ah_{\tau} \defeq \real Y(\tau X)$が定める制限ルート系も,ある$\alpha_{\tau} \in \ah_{\tau}^{*}$,$ \norm{\alpha_{\tau}} = 1 $により$\Sigma(\ge,\ah_{\tau}) = \{\pm\alpha_{\tau} \}$となる.任意の$\tau,\tau'\in \real$に対し,ある$k\in K$が存在して$k\cdot \alpha_{\tau}  = \alpha_{\tau'} $であるから,$\norm{\alpha} \defeq \norm{\alpha_{\tau}} $は$\tau\in \real$に依らない.

  
  
  % \bluetext{定数を確認せよ $A_{\alpha}\in \ha $を任意の$H\in \ha $に対して$B(H,A_{\alpha}) = \alpha(H) $を満たす元とすると, 任意の固定した$\tau \in \real$に対し,$Y(\tau X) $に対応する$\ge_{\tau} $の元は$\inlineequation[eq:corresp]{\dfrac{B(Y(\tau X), A_{\alpha}) }{\norm{A_{\alpha}}}
  % \begin{pmatrix}
  %   0 & 1 \\
  %   1 & 0 
  % \end{pmatrix}}
  % $である.}
  
  % $Y(\real X) $の非有界性より,適当に列$\{t_{n}\in \real\}_{n\in \nat} $,$t_n\underset{n\to \infty}{\longrightarrow} \infty$を取ると$0 <  \norm{Y(t_n X)} = s_n $,$\forall n\in \nat$かつ$s_n\to \infty$,$n\to \infty$とできる.% さらに$Z(t_n X) = r_n Z_n $,$Z_n\in \per{\ha}\cap \pe $,$\norm{Z_n} = 1 $,$r_n\in \real$と書くと,\Cref{thm:kob97}より$\abs{r_n}\to \infty$,$n\to \infty$である.$\norm{Z_n} = 1 $より,必要なら$\{t_{n}\in \real\}_{n\in \nat} $の部分列を取ることで$\lim_{n\to \infty}Z_n = Z_{\infty}\in \per{\ha}\cap \pe $,$\norm{Z_{\infty} } = 1$なる$Z_{\infty}$が存在する.

  % 必要ならpositive systemを取り替えることで$B(Y(t_n X), A_{\alpha}) > 0$と仮定する.このとき$B(Y(t_n X), A_{\alpha}) = s_n$であるから,\Cref{eq:corresp}より$\ge'\simeq \sulie(1,1) $の同型において$Y(t_n X)  $は$s_n\begin{pmatrix}
  %   0 & 1 \\
  %   1 & 0 
  % \end{pmatrix}$に対応する.

  % 以上の性質を満たす$\{t_n\}_{n\in \nat} $,$\{s_n\}_{n\in \nat}  $% ,$\{r_n \}_{n\in \nat} $,$\{Z_{n} \}_{n\in \nat} $,$Z_{\infty} $
  % に対して

  
  \textbf{ある$\alpha\in (\ha\cap\pe)^{*}\setminus\{0\} $が存在して$\Sigma(\ge,\ha\cap\pe) = \{\pm\alpha,\pm 2\alpha \}$なるとき}

  \Cref{cor:sub-lie-alg}により$X$と$\ha \cap\pe$を含む部分Lie環$\ge^* \subset \ge $で$\sulie(2,1) $に同型なものが存在する.$\ge^* $に対応する$G$の解析的部分群を$G^* $とし,その岩澤分解を$G^* = K^*A^*N^* $とする.このとき$e^{Z(t X)}\cdot o_K = e^{-Y(tX)}e^{tX}\cdot o_K\in G^*/K $であるから$Z(tX)\in \ge^* \cap \per{\ha}\cap \pe\subset \ge^*\cap\pe $であり,$Y(\real X) $の有界性の議論は全測地的的な部分Riemann多様体$G^*/K^* \subset G/K$に対して行えば良いことがわかる.したがって\Cref{prop:classical-rank-one}により$X\in\pe_{H,\bdd}$であることと$ X \in\{0\}\cup \pe\setminus\ha $が同値であることが言え,\Cref{thm:1216-main}が示された.
\end{npfwn}

    % 場合1と同様に,任意の固定した$\tau \in \real$に対し$\ah_{\tau} \defeq \real Y(\tau X)$が定める制限ルート系も,ある$\alpha_{\tau} \in \ah_{\tau}^{*}$,$ \norm{\alpha_{\tau}} = 1 $により$\Sigma(\ge,\ah_{\tau}) = \{\pm\alpha_{\tau}, \pm 2\alpha_{\tau} \}$となるが,$Y(\real X) $の非有界性より,ある$T\in \real$で$\abs{\alpha_{t}(Y(tX))} >  $\bluetext{埋めよ},$\forall t \geq T$なるものが存在する.
    
\begin{rem}
  \Cref{thm:1216-main}では$G$の線型性を仮定したが,この仮定は不要である.つまり,$G$を線型とは限らない実半単純Lie群としても\Cref{thm:1216-main}が成り立つ.

  なぜならば,$G$の中心$Z$ (離散群) の部分群$Z_0$で割ったときの$G/Z_0$は$G$と局所同型であるから,次の\Cref{lem:loc-isom}より$G$が線型の場合に帰着される.

  \begin{defi}\label{def:loc-isom}
    実半単純Lie群$G_1$,$G_2$が局所同型であることを,それぞれのLie環$\ge_1$,$\ge_2$の間の同型写像$\phi\colon \ge_1\to \ge_2$とそのCartan分解$\ge_i = \ka_i\oplus \pe_i $とKilling形式に対し,$\phi(\ka_1) = \ka_2 $,$\phi(\pe_1) = \pe_2 $かつ$\phi|_{\pe_1} \colon \pe_1\to \pe_2$が内積を保つことを言う.
  \end{defi}

  \begin{lem}\label{lem:loc-isom}
    $G_1$,$G_2$を\Cref{def:loc-isom}の意味で局所同型な連結実半単純Lie群とし,局所同型を与える同型写像を$\phi\colon \ge_1\to \ge_2$とする.$\phi\colon \ge_1\to \ge_2$に対応する$G_1$と$G_2$の準同型を$\Phi\colon G_1\to G_2$とする.

    $G_1$,$G_2$のCartan対合を$\Theta_1$,$\Theta_2$とする.閉部分群$H_1\subset G_1$,$H_2\subset G_2$は$\Theta_1H_1 = H_1$,$\Theta_2H_2 = H_2$を満たし,$\Phi(H_1) = H_2 $を満たすとする.

    このとき$X\in \pe_1$に対し,$Y_1(\real X) $が有界であることと,$Y_2(\real \phi(X)) $が有界であることは同値である.
  \end{lem}
  \begin{npfwn}{\Cref{lem:loc-isom}}
    $\phi(\ka_1) = \ka_2 $より$\Phi(K_1) = K_2 $であるから,$\bar{\Phi}\colon G_1/K_1\to G_2/K_2 $がwell-definedに定まる.

    このとき,任意の$X\in \pe_1$に対し,$\bar{\Phi}(e^X\cdot o_{K_1}) = e^{\phi(X)}\cdot o_{K_2} $であり,$\bar{\Phi}(e^{Y_1(X)}e^{Z_1(X)}\cdot o_{K_1}) = e^{\phi(Y_1(X))}e^{\phi(Z_1(X))}\cdot o_{K_2} $である.

    $\phi|_{\pe_1} \colon \pe_1\to \pe_2$が内積を保つことと$\phi(\ha_1) = \ha_2 $より,$\phi(Y(X))\in \ha_2\cap \pe_2 $,$\phi(Z(X))\in \per{\ha_2}\cap \pe_2 $であり,したがって\Cref{thm:kob89-lem6.1}より$Y_2(X) = \phi(Y_1(X)) $,$Z_2(X) = \phi(Z_1(X)) $である.

    $\phi|_{\pe_1} \colon \pe_1\to \pe_2$は内積を保つ写像であったから,$Y_1(\real X) $が有界であることと,$Y_2(\real \phi(X)) $が有界であることは同値である.
  \end{npfwn}
\end{rem}


\subsubsection{補足: \Cref{thm:1216-main}の微分幾何的側面}
\begin{defi}({\cite[Definition~1.3]{e72-1}})\label{def:visibility}
  $M$が完備かつ非正の断面曲率をもつ連結かつ単連結なRiemann多様体であるとき,$M$をHadamard多様体という.

  Hadamard多様体$M$がvisibility manifoldであるとは,任意の$ p\in M$と任意の$ \epsilon > 0$に対し,ある$r(p,\epsilon) >0 $が存在して,測地線$\gamma\colon [t_0, t_1]\to X $が任意の$ t\in [t_0, t_1]$に対し$r(p,\epsilon) \leq d_{M}(p, \gamma(t))$を満たすならば,$\measuredangle_{p}(\gamma(t_0), \gamma(t_1)) \leq \epsilon $であることである.
\end{defi}

後に示すように{\Poincare}円板はvisibility manifoldであるが,\Cref{lem:0106}よりその片鱗を見ることはできる.具体的には$ 0 <  \epsilon  < \frac{\pi}{2} $に対し$t_{\epsilon} \defeq \dfrac{1}{2}\inv{\sinh}\dfrac{1}{\abs{\tan \epsilon}} $とし,測地線$\gamma_{\epsilon}(s) = e^{t_{\epsilon} Y}e^{sZ}\cdot o_K $とすると,\Cref{lem:0106}より任意の$s_0,s_1\in \real$に対し$\measuredangle_{o_K}(\gamma_{\epsilon} (s_0), \gamma_{\epsilon} (s_1)) \leq \epsilon $である.この様子を図示すると\Cref{fig:visibility}のようになる.

\begin{figure}[H]
  \centering
  % \raggedleft
  % \raggedrightp
  % \includegraphics[scale=0.08]{../graph/riem-su11.png}
  \includegraphics[scale=0.3]{../graph/visibility-2.pdf}
  \caption{visibility manifoldのイメージ}
  \label{fig:visibility}
\end{figure}

\begin{defi}({\cite[p.~202]{bh99}})
  $M$をHadamard多様体,$\isom(M)$は$M$の等長同型群とする.あるコンパクト集合$C\subset M$で$ M = \bigcup\{f(C)\mid f\in \isom(M) \}  $なる$C$が存在するとき,$M$はcocompactであるという.
\end{defi}

\begin{thm}{({\cite[p.~296, 9.33~Theorem]{bh99}, 原典: \cite[Theorem~4.1]{e72-2}})}\label{thm:visibility-and-rank}  
  cocompactなHadamard多様体$M$に対し,次は同値である.
  % \vspace{-1em}
  \begin{enumerate}
    \renewcommand{\labelenumi}{(\roman{enumi})}
  \item $M$はvisibility manifoldである.
  \item 全測地的な部分Riemann多様体$M'\subset M$で$\real^2$と等長同型なものが存在しない.
  \end{enumerate}
\end{thm}

ここで\Cref{thm:non-positivity}よりRiemann対称空間$G/K\simeq \pe$はcocompactなHadamard多様体である.
\begin{thm}(\cite[p.~241, Theorem~3.1]{hel01})\label{thm:non-positivity}
  $G$を実半単純Lie群,$K$を$\ka$をLie 環に持つ$G$の部分Lie 群とするとき,$G/K$は$\ge$のKilling形式から誘導されるRiemann多様体の構造を持ち,その断面曲率は至るところ非正である.
\end{thm}

次の\Cref{thm:rank-of-symm-sp}より,$G/K$に対して\Cref{thm:visibility-and-rank}の (ii) は$G$の実階数が1以下であることと同値である.

\begin{thm}({\cite[p.~245, Proposition~6.1]{hel01}})\label{thm:rank-of-symm-sp}
  $G/K$を\Cref{thm:non-positivity}の設定の通りとする.

  $G$の実階数と$G/K$内の全測地的で平坦な部分多様体の最大次元は一致する.
\end{thm}

特に$G$の実階数が1の場合$G/K$はvisibility manifoldである.よって\Cref{thm:1216-main}の設定のもとでは,角度を用いた$G = \SU(1,2) $,$H= \SO(1,1)$の場合の$\pe_{H,\bdd} = \{0\}\cup\pe\setminus\ha $の証明である\Cref{rem:su11-by-angle}と全く同様に,$X\in \pe\setminus\ha $の場合に$Y(\real X) $が非有界であると仮定して矛盾を示す論法で$\pe_{H,\bdd} =\{0\}\cup\pe\setminus\ha $が示される.

\subsection{$\dim H =1$かつ$\ha$の基底が generic の場合}
\begin{thm}\label{thm:0106-main}
    $G$を非コンパクト実半単純Lie 群とするとき,
    $\ha = \real Y$かつ$\ze_{\ge}(Y) = \emm + \ah$ならば \Cref{yosou:1121} が成り立つ.
\end{thm}


\begin{lem}\cite[Corollary~I.2.17]{borel-ji}\label{lem:bj-1.2.17}
  
  $\ah_{P}^{+}(\infty)\ni W\1to1arrow [e^{tW}\cdot o_K]\in G/K(\infty) $により$\ds G/K(\infty) = \bigsqcup_{P:\text{ psg}}\ah_{P}^{+}(\infty) $である.

  以下,この主張を用いて$G/K(\infty) $の元と$\ah_{P}^{+}(\infty)$,あるいは$\bar{\ah_{P}^{+}(\infty)} $の元を同一視する.
\end{lem}
\begin{lem}\cite[Proposition~I.2.6, Corollary~I.2.17]{borel-ji}\label{lem:bj-1.2.6}

  $G/K(\infty) $の任意の点の固定部分群は$G$ではない放物型部分群であり,放物型部分群$P$を固定部分群として持つ$G/K(\infty) $の元は$\bar{\ah_{P}^{+}(\infty)} $に一致する.
  
\end{lem}



\begin{lem}(\cite[Proposition~6.7]{eo73}, \cite[2.8~Lemma]{bbe85})\label{lem:axis-isometry}
  
  任意の$p\in G/K(\infty) $と点列$\{t_n\}_{n\in \nat},\; t_n\to \infty $,$n\to \infty$に対し,$\{e^{t_nY}\cdot p\}_{n\in \nat} $の任意の集積点は$[\ah]$の元である.
\end{lem}



\begin{pfwn}{\Cref{lem:axis-isometry}}

    必要ならば部分列に移って$\{e^{t_n Y}\cdot p \}_{n\in \nat} $は収束すると仮定する.
  
  任意の$I\subsetneq \Delta(\ge,\ah) $に対し,$P_0\subset P_I $とBruhat分解$G= \bigsqcup_{w\in W}\bar{N}wP_0 $より$G = \bigcup_{w\in W}\bar{N} k_wP_{I} $である.したがって$p \in \ah_{P}^{+} (\infty)$なる放物型部分群$P$を取ると,ある$I$と$\bar{n}\in \bar{N} $,$w\in \cal{W} $が存在して,$P = \bar{n}k_wP_{I}\inv{k_w}\inv{\bar{n}} $と書ける. 以下,$P_{I}'\defeq k_wP_{I}\inv{k_w}$とする.

  $e^{t_n Y}\cdot p $の固定部分群は$P_{n} \defeq e^{t_n Y}Pe^{-t_n Y} $であり,【$k_n\in K $を$K/(K\cap P'_{I})\simeq G/P'_{I} $という同一視 % ($G=KP_{I}' $を用いた) 
  のもとで,$k_n(K\cap P_{I}') = e^{t_n Y}\bar{n}e^{-t_n Y}P_{I}' $かつ$\norm{-}_{\theta} $から定まる$K/(K\cap P'_{I})$の計量に対して最も原点$\id_G(K\cap P'_{I}) $に近い元とすると,】\footnote{この$k_n$はもう少し単純に表せる気がします.} $\inv{k_w}e^{t_n Y}k_w\in P_{I} $より$P_n =  k_nP_{I}'\inv{k_n} $である.\Cref{lem:bj-1.2.6}より$e^{t_n Y}\cdot p\in \bar{\ah_{P_n}^{+}(\infty)} $であり$\bar{\ah_{P_n}^{+}(\infty)} \subset \Ad(k_n)\ah_{P_{I}'}  $より,ある一意的な$W_n\in \ah_{P_{I}'} $,$\norm{W_n}_{\theta} = 1 $が取れて,$e^{t_n Y}\cdot p $と$\ds\dfrac{\Ad(k_n)W_n}{\norm{\Ad(k_n)W_n}_{\theta}}$が\Cref{lem:bj-1.2.17}の意味で対応する.

  
  必要ならば再度部分列に移ると$\norm{W_n}_{\theta}=1 $より$\lim_{n\to \infty}W_n = \exists W \in \ah_{P_{I}'} $であり,$\lim_{n\to \infty}e^{t_n Y}\bar{n}e^{-t_n Y} = \id_G$であることと,$k_n$と原点の距離の最短性より$ \lim_{n\to \infty}k_n = \id_G $であるから$\lim_{n\to\infty} \dfrac{\Ad(k_n)W_n}{\norm{\Ad(k_n)W_n}_{\theta}}= W  $である.

  収束する点列の部分列はもとの収束先に収束するから$\lim_{n\to \infty} e^{t_n Y}\cdot p$と$W\in \ah_{P_{I}'}$が\Cref{lem:bj-1.2.17}の意味で対応し,$\ah_{P_{I}'}\subset \Ad(k_w)\ah_I\subset \Ad(k_w)\ah\subset \ah $より$\lim_{n\to \infty} e^{t_n Y}\cdot p\in [\ah] $である.
  % したがって任意のに対して$\ds \lim_{n\to\infty}\lbig\|W - \dfrac{\Ad(e^{t_n Y}\bar{n}e^{-t_n Y})W}{\norm{\Ad(e^{t_n Y}\bar{n}e^{-t_n Y})W}_{\theta}} \rbig\|_{\theta} = 0 $と\Cref{rem}より,
  

  したがって\Cref{lem:axis-isometry}が示された.

\end{pfwn}

\begin{pfwn}{\Cref{thm:0106-main}}

  \Cref{yosou:1121} について,$\pe_{H,\bdd}\subset \{X\in \pe\mid [X_1,X_2] \neq 0\text{ or } X_1 =0 \}$は明らかであるから,\\
  $\pe_{H,\bdd}\supset \{X\in \pe\mid [X_1,X_2] \neq 0\text{ or } X_1 =0 \}$を示せば良い.さらに$X_1 = 0\implies X\in \pe_{H,\bdd} $も明らかであるから,今示すべきは,$X\in \pe \st [X_1, X_2] \neq 0\implies X\in \pe_{H,\bdd} $である.

  % $\ze_{\ge}(Y) $: amenable より,ある極小放物型部分 Lie 環$\qu_{0} = \emm_0 + \ah_0 + \enn_{0} $が存在して,$Y\in \ah^{+}_0 $である.さらに
  $[X_1, X_2]\neq 0$より,$ [X, Y] = [X_2, Y]\neq 0 $であるから,$X\nin \ah $である.

  ここで$\norm{Y(\real X)} $が非有界であると仮定する.このとき$X\nin \ha$であるから \cite[Lemma~5.4]{kobayashi97} より$\norm{Z(\real X)} $も非有界である.必要なら$Y$の符号を調整し,列$\{t_n\geq 0\}_{n\in\nat} $であって$s_n\to \infty$,ただし$Y(t_nX) = s_{n}Y $,$s_{n}\in \real $,かつ$e^{Z(t_nX)}\cdot o_K\to [Z]\in G/K(\infty) ,\; Z\in \per{\ha}\cap \pe $,$n\to \infty$なるものが取れる ($e^{Z(t_nX)}\cdot o_K $の収束は$\bar{G/K}^{c} $のコンパクト性による).

  ここで$U(\epsilon, r)\defeq \bigcup_{x\in [\ah]} C(x,\epsilon,r) $とすると,$U(\epsilon, r) $は$[\ah]$の任意の点の近傍であり,\Cref{lem:axis-isometry} より,任意の$r, \epsilon > 0$に対して$\{s_n\}_{n\in\nat} $の部分列を取り直せば,$\forall n> N$,$e^{s_nY}[Z] \cdot o_K\in U(r,\epsilon) $である.また$G\leftaction \bar{G/K}^{c}  $の連続性より,ある$C([Z],\epsilon',  r'  ) $を取ると,$e^{s_nY}\cdot C([Z],\epsilon',  r'  )\subset U(r,\epsilon)  $である.$e^{Z(t_nX)}\cdot o_K\to [Z]\in G/K(\infty) ,\; Z\in \per{\ha}\cap \pe $,$n\to \infty$であるから,必要ならばさらに$N$を大きく取り直すことで$\forall n > N$,$e^{Z(t_nX)}\cdot o_K\in C([Z],\epsilon',  r'  )$,したがって$e^{s_n Y}e^{Z(t_n Z)}\cdot o_K\in U(\epsilon, r) $である.
  
  しかし$e^{s_n Y}e^{Z(t_n X)}\cdot o_K= e^{t_n X}\cdot o_K $と$A\cdot o_K$上の$o_K$を通る測地線のなす角の最小値は$X$と$\ah\setminus\{0\} $のなす角で,非零であるから矛盾する.
  % $\psi(X,\ah)\neq 0 $
  
\end{pfwn}



\section*{謝辞}
\addcontentsline{toc}{section}{謝辞}

本研究および修士課程全体において常に洞察に富むご助言と丁寧なご指導を賜った指導教員の小林俊行教授に深謝の意を表する.% 小林先生のご指導なしには本論文はあり得なかった.
また,文献の情報から数学的な議論にわたり様々なご助言をくださった,修了生も含む小林研究室のみなさまにも心より感謝する.特に小林研究室の田内大渡氏には幾度か議論いただいたことに御礼申し上げたい.

最後に,学部時代からセミナーに付き合ってくださり,ときに精神的にも支えてくださった友人と家族に感謝の意を表して謝辞とする.

\begin{comment}
  多くのご指導とご協力を賜りました小林俊行先生に深謝の意を表します.粘り強く丁寧なご指導のおかげで本論文は完成いたしました.
  また,論文執筆の際にご助言や相談に乗っていただいた,研究室の各位そして同期の皆様にも感謝いたします.
  最後に,温かく見守り支援してくださった友人と家族に感謝の意を表して謝辞といたします.




  First of all, I would like to show my profound gratitude to professor Toshiyuki Kobayashi, for his constant support and encouragement, who introduced me to this beautiful field. His strict and patient guidance broadened my knowledge, making me grow up as a mathematician. This master dissertation has been possible thanks to him.
  I am also deeply thankful to all the members of Kobayashi’s lab; specially to K. Kannaka, T. Satomi, Y. Inoue, R. Fujita and T. Okuda, who are always supporting me and keeping me active with interesting questions and debates.
  The support of the office members of the faculty was also remarkable, specially the case of A. Nakamura; who constantly encouraged me these last two years.
  I would also like to express my appreciation to I. Vald ́es S ́anchez, J. M. Gonz ́alez Vega, S. Ishii and R. S ́anchez Molero for their continuous encouragement and understanding; they are always keeping me motivated.
  Last but not least, special thanks to the Japanese Ministry of Education, Culture, Sports, Science and Technology (MEXT), for their financial support throughout the last two years.
\end{comment}

\begin{thebibliography}{99}
  \addcontentsline{toc}{section}{参考文献}
  % \bibitem[Benoist 09]{benoist-five-lectures} Y. Benoist, \textit{Five lectures on lattices in semisimple Lie groups}, S\'emin. Congr., Tome 18, 2009, pp. 117--76, Retrieved July 24, 2020 from \url{https://www.imo.universite-paris-saclay.fr/~benoist/prepubli/04lattice.pdf}\footnote{正式に出版された版 \url{https://smf.emath.fr/publications/cinq-cours-sur-les-reseaux-des-groupes-de-lie-semisimples} には大学経由でもアクセスできませんでした.}
\bibitem[Ber88]{ber88} J. N. Bernstein, \textit{On the support of Plancherel measure}, J. Geom. Phys., \textbf{5}, (1988), 663--710.
  % \bibitem[Bernstein--Reznikov]{ber-rez} J. Bernstein and A. Reznikov, \textit{Analytic continuation of representations and estimates of automorphic forms}, Ann. of Math. (2), Vol. 150, Issue 1, 1999, pp. 329--52
  % \bibitem[Borel]{borel276} A. Borel, Repr\'esentations de groupes localement compacts, Lecture Notes in Mathematics 276, Springer, 1972
\bibitem[BBE85]{bbe85} W. Ballmann, M. Brin and P. Eberlein, \textit{Structure of manifolds of nonpositive curvature. I}, Ann. of Math. (2),  \textbf{122}, (1985), 171--203.
\bibitem[BH99]{bh99} M.~R.~Bridson and A.~Haefliger, Metric Spaces of Non-Positive Curvature, Grundlehren der mathematischen Wissensschaften, Vol.~319, Springer, 1999.
\bibitem[Borel--Ji]{borel-ji} A. Borel and L. Ji, Compactifications of Symmetric and Locally Symmetric Spaces, Mathematics: Theory \& Applications,  Birkhäuser  Boston, 2006.
  % \bibitem[Bourbaki Int. 7 et 8]{bour-int-7-and-8} N. Bourbaki, Int\'egration, Chaptires 7 et 8, \'El\'ements de math\'ematique, Springer, 2007
  % \bibitem[Bruhat 56]{bruhat56} F. Bruhat, \textit{Sur les repr\'esentations induites des groupes de Lie}, Bull. Soc. Math. France, \textbf{84}, 1956, pp. 97--205
  % \bibitem[Casselman]{casselman-dix-mal} William (Bill) Casselman, Essays on representations of real groups---The theorem of Dixmier \& Malliavin, Last revised January 24, 2016, Retrieved March 5, 2020 from \url{https://www.math.ubc.ca/~cass/research/pdf/Dixmier-Malliavin.pdf}
  % \bibitem[Conway]{conway} J. B. Conway, A Course in Functional Analysis---Second Edition, GTM 96, Springer, 2007
  % \bibitem[Dieudonn\'e]{dieudonne} J. Dieudonn\'e, Fundations of modern analysis---enlarged and corrected printing, international edition, Academic Press, 1969
\bibitem[Ebe72a]{e72-1} P.~Eberlien, \textit{Geodesic Flows on Negatively Curved Manifolds I}, Ann.~of~Math.~(2), \textbf{95}, (1972), 492--510.
\bibitem[Ebe72b]{e72-2} P.~Eberlien, \textit{Geodesic Flow in Certain Manifolds without Conjugate Points}, Trans.~Amer.~Math.~Soc., \textbf{167}, (1972), 151--70.
\bibitem[EO73]{eo73} P. Eberlein and B. O'Neill, \textit{Visibility Manifolds}, Pacific J. Math., \textbf{46}, (1973), 45--109.
  % \bibitem[Gromov \& Pansu]{gro-pan} M. Gromov and P. Pansu, \textit{Rigidity of lattices: An introduction}, In: P. de Bartolomeis and F. Tricerri eds., Geometric Topology: Recent Developments, Lecture Notes in Mathematics, vol. 1504, Springer, 1991
  % \bibitem[Halmos]{halmos} Paul. R. Halmos, Measure Theory, GTM 18, Springer, 1974
\bibitem[Hel84]{hel84} S. Helgason, Groups and Geometric Analysis---Integral Geometry, Invariant Differential Operators, and Spherical Functions, Mathematical Surveys and Monographs, Vol. 83, AMS, 1984.
\bibitem[Hel01]{hel01} S. Helgason, Differential Geometry, Lie Groups, and Symmetric Spaces, GSM, Vol. 34, AMS, 2001.
  % \bibitem[Jenkins 73]{jenkins73} J. W. Jenkins, \textit{Growth of connected locally compact groups}, J. Funct. Anal., Vol. 12, Issue 1, 1973, pp. 113--27
  % \bibitem[Kassel--Kobayashi 16]{kk16} F. Kassel and T. Kobayashi, \textit{{\Poincare} series for non-Riemannian locally symmetric spaces}, Adv. Math., Vol. 287, 2016, pp. 123--236
  % \bibitem[Knapp 96]{knapp-beyond} A. W. Knapp, Lie Groups Beyond an Introduction, Progress in Mathematics, Vol. 140, Birkhäuser, 1996
\bibitem[KK16]{kk16} F. Kassel and T. Kobayashi, \textit{{\Poincare} series for non-Riemannian locally symmetric spaces}, Adv. Math., \textbf{287}, (2016), 123--236.
\bibitem[Kob89]{kob89} T. Kobayashi, 
  \textit{Proper action on a homogeneous space of reductive type},
  Math. Ann., \textbf{285}, (1989), 249--263.  
  % \bibitem[Kobayashi 96]{kobayashi96} T. Kobayashi, \textit{Criterion for proper actions on homogeneous spaces of reductive groups}, J. Lie Theory, Vol. 6, 1996, pp. 147--63
\bibitem[Kob97]{kob97} T. Kobayashi, \textit{Invariant mesures on homogeneous manifolds of reductive type}, J. Reine Angew. Math., \textbf{1997}, (1997), 37--54.
\bibitem[小林95]{kobayashi95} 小林俊行,球等質多様体上の調和解析入門,第3回整数論サマースクール `等質空間と保型形式' 所収,佐藤文広 編,長野,(1995),22--41.
  % \bibitem[Kobayashi 07]{kobayashi07} T. Kobayashi, \textit{A generalized Cartan decomposition for the double coset space $(U(n_1)\times U(n_2)\times U(n_3))\backslash U(n)/(U(p)\times U(q)) $}, J. Math. Soc. Japan, Vol. 59, No. 3, 2007, pp. 669--691
\bibitem[小林--大島]{kando} 小林俊行・大島利雄,リー群と表現論,岩波書店,2005.
\bibitem[Lee18]{lee2018} J. M. Lee, Introduction to Riemannian Manifolds---Second Edition, GTM 176, Springer, 2018.
  % \bibitem[Reed--Simon I]{reed-simon-1} M. Reed \& B. Simon, Methods of Modern Mathematical Physics---I: Functional Analysis, Academic Press, 1972
  % \bibitem[Sasaki 10]{sasaki10} A. Sasaki, \textit{A characterization of non-tube type Hermitian symmetric spaces by visible actions}, Geom. Dedicata, Vol. 145, 2010, pp. 151--8 
  % \bibitem[辰馬]{tatsuma} 辰馬伸彦,位相群の双対定理,紀伊國屋数学叢書 32,紀伊國屋書店,1994
  % \bibitem[Moore 68]{moore} R. T. Moore, Measurable, continuous and smooth vectors for semigroups and group representations, Mem. Amer. Math. Soc., no. 78, 1968
  % \bibitem[Morris 15]{morris} D. W. Morris, \textit{Introduction to Arithmetic Groups}, arXiv:math/0106063v6, Published by Deductive Press, 2015
  % \bibitem[Varadarajan]{varadarajan} V. S. Varadarajan, Lie Groups, Lie Algebras, and Their Representations, GTM 102, Springer, 1984
  % \bibitem[Wallach I]{wallach1} N. R. Wallach, Real Reductive Groups I, Pure and Applied Mathematics, Vol. 132, Academic Press, 1988
  % \bibitem[吉田]{yoshida} 吉田耕作,復刊 超函数論,共立出版,2009
\bibitem[Yos37]{yos37} K.~Yosida, \textit{A problem concerning the second fundamental theorem of Lie}, Proc.~Imp.~Acad., \textbf{13}, (1937), 152--155.
\bibitem[Yos38]{yos38} K.~Yosida, \textit{A Theorem concerning the Semi-Simple Lie Groups}, \\Tohoku~Mathematical~Journal, First Series, \textbf{44}, (1938), 81--84.

  %%% Local Variables:
  %%% mode: latex
  %%% TeX-master: "okuda-master-thesis"
  %%% End:

\end{thebibliography}

\end{document}
